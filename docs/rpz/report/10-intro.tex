\chapter*{ВВЕДЕНИЕ}
\addcontentsline{toc}{chapter}{ВВЕДЕНИЕ}

Сфера информационных технологий (IT) испытывает нехватку кадров. Каждый год университеты страны выпускают специалистов, но их
количество не удовлетворяет спрос.

Из-за популярности it-специальностей, в последние годы набирают популярность курсы, на которых за небольшой промежуток времени
готовят специалистов в различных направлениях.

На таких курсах существует менторинг - программа поддержки студентов. Ментор - это профессионал, обладающий компетенциями в своей сфере деятельности, источник знаний и ответов. Он помогает развиваться своему подопечному — менти. 

Ментор преследует цель - поделиться собственным опытом с менти. Но на образовательных курсах ментор помогает, в основном, в рамках программы, на которой построен курс. 

Другое направление менторинга - составление индивидуального плана развития.
На программах при образовательных курсах программа менторства не покрывается исчерпывающе, доля персонального развития недостаточна.
Индивидуальный подход более действеннен для людей, имеющих некоторый опыт в изучаемой сфере и желающих повысить свой уровень. Например, младший инженер-программист хочет повысить свой уровень до старшего. Для этого ему нужен человек, который сможет составить план, состоящий из конкретных шагов, при достижении которых, уровень компетенции возрастет до желаемого уровня.  

Ментор - специалист, работающий в индустрии и знающий, какие навыки востребованы прямо сейчас, как повысить квалификацию.
Этот человек способен составить программу и в короткие сроки подготовить менти к будущей работе. Другое направление - повышение уровня квалификации уже работающего специалиста.

Для того, чтобы найти наставника и составить план развития, нужна система, предоставляющая такую возможность. 

Реализация системы требует наличия базы данных и методов взаимодействия с ней.

Цель работы -- реализовать базу данных для приложения по поиску наставника для изучения информационных технологий.

Чтобы достигнуть поставленной цели, требуется решить следующие задачи:

\begin{itemize}
    \item проанализировать варианты модели базы данных и выбрать подходящий вариант для решения задачи;
    \item спроектировать базу данных, описать ее сущности и связи;
    \item реализовать интерфейс для доступа к базе данных;
    \item реализовать программное обеспечение, которое позволит получить доступ к данным.
\end{itemize}