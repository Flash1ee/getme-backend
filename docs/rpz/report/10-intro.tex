\chapter*{ВВЕДЕНИЕ}
\addcontentsline{toc}{chapter}{ВВЕДЕНИЕ}

Сфера информационных технологий (IT) испытывает нехватку кадров. Каждый год университеты страны выпускают специалистов, но их
количество не удовлетворяет спрос.

Из-за популярности it-специальностей, в последние годы набирают популярность курсы, на которых за небольшой промежуток времени
готовят специалистов в различных направлениях.

На таких курсах существует менторинг - программа поддержки студентов. Ментор - это профессионал, обладающий компетенциями в своей сфере, источник знаний и ответов. Он помогает развиваться в профессиональной жизни своему подопечному — менти. 

Ментор преследует цель - поделиться собственным опытом с менти. Но на образовательных курсах ментор помогает, в основном, в рамках программы, на которой построен курс. 

Другое направление менторинга - составление индивидуального плана развития,  в программах, описанных выше, не покрывается в исчерпывающе. Индивидуальный подход более действеннен для людей, имеющих некоторый опыт в изучаемой области и желающих повысить свой уровень. Например, младший инженер программист хочет повысить свой уровень до старшего. Для этого ему нужен человек, который сможет составить план, состоящий из конкретных шагов, при достижении которых, уровень компетенции возрастет до желаемого уровня.  

Ментор - специалист, работающий в индустрии и знающий, какие навыки востребованы прямо сейчас. Этот человек
способен составить программу и в короткие сроки подготовить менти к будущей работе или повысить уровень квалификации уже работающим специалистам.

Для того, чтобы найти наставника и составить план развития, нужно приложение, предоставляющее данную функциональность. 

Реализация приложения требует наличия базы данных и методов взаимодействия с ней.

Цель работы -- реализовать базу данных для приложения по поиску наставника для изучения информационных технологий.

Чтобы достигнуть поставленной цели, требуется решить следующие задачи:

\begin{itemize}
    \item проанализировать варианты представления данных и выбрать подходящий вариант для решения задачи;
    \item проанализировать системы управления базами данных и выбрать подходящую систему для хранения данных;
    \item спроектировать базу данных, описать ее сущности и связи;
    \item реализовать интерфейс для доступа к базе данных;
    \item реализовать программное обеспечение, которое позволит получить доступ к данным.
    \item исследовать производительность базы данных.
\end{itemize}