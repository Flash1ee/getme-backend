\chapter{Конструкторская часть}
%В данном разделе рассмотрено проектирование базы данных и приложения.

\section{Проектирование базы данных}
Для основной логики приложения была спроектирована база данных, представленная в виде ER-модели, в нотации Crow's Foot.  

\imgw{main-erd}{h!}{1.0\textwidth}{ER диаграмма основной логики приложения}


База данных будет состоять из следующих сущностей:
\begin{enumerate}
\item Users - пользователи;
\item Plans - планы развития;
\item Task - задачи из плана развития;
\item Skills - изучаемые навыки;
\item SimpleAuth - данные аутентификации вида логин-пароль;
\item TelegramAuth - данные о телеграмме пользователя;
\item Status - статус выполнения задачи;
\item Панель администратора - набор сущностей панели администратора
\item Sessions - хранение сессий пользователей
\end{enumerate}

\newpage
\noindent\textbf{Сущность Users}\\
Сущность Users содержит информацию о пользователях:
\begin{table}[!ht]
    \caption{Описание полей таблицы \texttt{users}}
    \label{tbl:users}
    \begin{center}
        \begin{tabular}{|p{0.2\textwidth}p{0.6\textwidth}|}
            \hline
            \textbf{Поле} & \textbf{Значение} \\\hline
            \texttt{id} & Идентификатор пользователя, уникальный. \\\hline
            \texttt{first\_name} & Имя пользователя в системе\\\hline
            \texttt{last\_name} & Фамилия пользователя \\\hline
            \texttt{nickname} & Псевдоним, короткая альтернатива имени \\\hline
            \texttt{about} & Информация о пользователе \\\hline
            \texttt{is\_searchable} & Признак отоюражения пользователя в поиске \\\hline
            \texttt{tg\_tag} & Псевдоним в Telegram \\\hline
            \texttt{created\_at} & Время регистрации пользователя\\\hline
            \texttt{updated\_at} & Время последнего редактирования информации о пользователе \\\hline
        \end{tabular}
    \end{center}
\end{table}

\noindent\textbf{Сущность Task}\\
Сущность Task содержит информацию о задачах, которые прикреплены к плану:
\begin{table}[!ht]
    \caption{Описание полей таблицы \texttt{plans}}
    \label{tbl:task}
    \begin{center}
        \begin{tabular}{|p{0.2\textwidth}p{0.6\textwidth}|}
            \hline
            \textbf{Поле} & \textbf{Значение} \\\hline
            \texttt{id} & Идентификатор задачи, уникальный. \\\hline
            \texttt{name} & Название задачи. \\\hline
            \texttt{description} & Описание задачи \\\hline
            \texttt{deadline} & Крайний срок выполнения \\\hline
            \texttt{status} & Статус задачи \\\hline
            \texttt{plan\_id} & Идентификатор плана развития, к которому привязана задача \\\hline
            \texttt{created\_at} & Время создания задачи\\\hline
        \end{tabular}
    \end{center}
\end{table}


\newpage
\noindent\textbf{Сущность Plans}\\
Сущность Plans содержит информацию о планах развития:
\begin{table}[!ht]
    \caption{Описание полей таблицы \texttt{plans}}
    \label{tbl:plans}
    \begin{center}
        \begin{tabular}{|p{0.3\textwidth}p{0.6\textwidth}|}
            \hline
            \textbf{Поле} & \textbf{Значение} \\\hline
            \texttt{id} & Идентификатор плана, уникальный. \\\hline
            \texttt{name} & Название плана развития. \\\hline
            \texttt{about} & Описание плана развития\\\hline
            \texttt{is\_active} & Признак выполнения плана \\\hline
            \texttt{progress} & Процент выполнения плана \\\hline
            \texttt{mentor\_id} & Идентификатор ментора, создавшего план \\\hline
            \texttt{mentee\_id} & Идентификатор менти, обучающемуся по плану \\\hline
            \texttt{created\_at} & Время создания плана \\\hline
            \texttt{updated\_at} & Время последнего изменения информации о плане \\\hline
        \end{tabular}
    \end{center}
\end{table}

\noindent\textbf{Сущность Skills}\\
Сущность Skills содержит информацию о навыках, существующих в системе:
\begin{table}[!ht]
    \caption{Описание полей таблицы \texttt{skills}}
    \label{tbl:skills}
    \begin{center}
        \begin{tabular}{|p{0.2\textwidth}p{0.6\textwidth}|}
            \hline
            \textbf{Поле} & \textbf{Значение} \\\hline
            \texttt{name} & Название навыка. \\\hline
            \texttt{color} & Цвет навыка (для бейджа) \\\hline
        \end{tabular}
    \end{center}
\end{table}

\newpage
\noindent\textbf{Сущность Status}\\
Сущность Status содержит информацию о навыках, существующих в системе:
\begin{table}[!ht]
    \caption{Описание полей таблицы \texttt{status}}
    \label{tbl:status}
    \begin{center}
        \begin{tabular}{|p{0.2\textwidth}p{0.6\textwidth}|}
            \hline
            \textbf{Поле} & \textbf{Значение} \\\hline
            \texttt{name} & Текстовое название статуса. \\\hline
            \texttt{color} & Цвет статуса (для бейджа) \\\hline
        \end{tabular}
    \end{center}
\end{table}

\noindent\textbf{Сущности Авторизации}\\
В приложении есть два способа авторизации: через мессенджер Telegram\cite{telegram} и через пару логин-пароль.  
Сущность SimpleAuth содержит данные авторизации пользователей через логин-пароль:
\begin{table}[!ht]
    \caption{Описание полей таблицы \texttt{users\_simple\_auth}}
    \label{tbl:users_simple_auth}
    \begin{center}
        \begin{tabular}{|p{0.3\textwidth}p{0.6\textwidth}|}
            \hline
            \textbf{Поле} & \textbf{Значение} \\\hline
            \texttt{id} & Идентификатор записи. \\\hline
            \texttt{login} & Логин пользователя. \\\hline
            \texttt{encrypted\_password} & Зашифрованнный пароль \\\hline
            \texttt{user\_id} & Идентификатор пользователя \\ & к которому привязаны данные авторизации \\\hline
        \end{tabular}
    \end{center}
\end{table}

Сущность TelegramAuth содержит данные авторизации пользователей через мессенджер Telegram:
\begin{table}[!ht]
    \caption{Описание полей таблицы \texttt{users\_simple\_auth}}
    \label{tbl:telegram_auth}
    \begin{center}
        \begin{tabular}{|p{0.2\textwidth}p{0.6\textwidth}|}
            \hline
            \textbf{Поле} & \textbf{Значение} \\\hline
            \texttt{tg\_id} & Имя пользователя в telegram. \\\hline
            \texttt{last\_auth} & Время последнего входа в приложение через telegram. \\\hline
            \texttt{user\_id} & Идентификатор пользователя, к которому привязаны данные авторизации \\\hline
            \texttt{created\_at} & Дата первого входа в систему через telegram \\\hline
        \end{tabular}
    \end{center}
\end{table}
\newpage

\noindent\textbf{Сущности панели администратора}\\
Для администрирования базы данных, поддержания ролевой модели будет создан ряд сущностей.
ER-модель сущностей, в нотации Crow’s Foot, представлена ниже:
\imgw{admin-erd}{h!}{1.0\textwidth}{ER диаграмма сущностей панели администратора}

\newpage
\noindent\textbf{Таблицы для миграций базы данных}\\
Миграции базы данных - это система контроля версий базы данных.
Для обновления базы данных необходимо определить сущность, которая будет контролировать текущую версию.

Сущность SchemaMigrations содержит данные о текущей версии базы данных:
\begin{table}[!ht]
    \caption{Описание полей таблицы \texttt{schema\_migrations}}
    \label{tbl:schema_migrations}
    \begin{center}
        \begin{tabular}{|p{0.2\textwidth}p{0.6\textwidth}|}
            \hline
            \textbf{Поле} & \textbf{Значение} \\\hline
            \texttt{version} & Номер текущей версии БД \\\hline
            \texttt{dirty} & Признак успешного обновления БД \\\hline
        \end{tabular}
    \end{center}
\end{table}

\section{Проектирование базы данных сессий}
Для отслеживания "состояния" между клиентом и сервером необходимо хранить данные, которые идентифицирует конкретного пользователя. Как говорилось выше, для таких задач используют in-memory СУБД.  
Такие базы данных не имеют одной структуры (как реляционные), поэтому можно лишь высокоуровнево спроектировать базу.

Для хранения пользовательские сессий требуются следующие поля:
\begin{table}[!ht]
    \caption{Описание полей таблицы \texttt{sessions}}
    \label{tbl:sessions}
    \begin{center}
        \begin{tabular}{|p{0.2\textwidth}p{0.6\textwidth}|}
            \hline
            \textbf{Поле} & \textbf{Значение} \\\hline
            \texttt{session\_id} & Идентификатор сессии, уникальный \\\hline
            \texttt{user\_id} & Идентификатор пользователя, которому принадлежит сессия \\\hline
        \end{tabular}
    \end{center}
\end{table}
\newpage
\section{Ограничения, связи между сущностями, целостность данных}
Для избежания дублирования данных сущности связаны между собой через внешние ключи.

\noindent\textbf{Внешние ключи}

\begin{itemize}
\item В Таблице telegram\_auth поле user\_id ссылается на поле id таблицы users;
\item В Таблице users\_simple\_auth поле user\_id ссылается на поле id таблицы users;
\item В Таблице plans 
	\begin{itemize}
		\item поле mentee\_id ссылается на поле id таблицы users;
		\item поле mentor\_id ссылается на поле id таблицы users;
	\end{itemize}
\item В Таблице status поле color ссылается на поле name таблицы color;
\item В Таблице task 
	\begin{itemize}
		\item поле plan\_id ссылается на поле id таблицы plans;
		\item поле status ссылается на поле name таблицы status;
	\end{itemize}
\item В таблице skills поле color ссылается на поле name таблицы color;
\item В Таблице offers 
	\begin{itemize}
		\item поле mentee\_id ссылается на поле id таблицы users;
		\item поле mentor\_id ссылается на поле id таблицы users;
		\item поле skill\_name ссылается на поле name таблицы skills;
	\end{itemize}
\end{itemize}
Помимо внешних ключей, в базе данных присутствуют связи типа many-to-many, реализующиеся через промежуточные таблицы.  

Для хранения пользовательские связей навык-пользователь используется таблица users-skills:
\begin{table}[!ht]
    \caption{Описание полей таблицы \texttt{users\_skills}}
    \label{tbl:users-skills}
    \begin{center}
        \begin{tabular}{|p{0.2\textwidth}p{0.6\textwidth}|}
            \hline
            \textbf{Поле} & \textbf{Значение} \\\hline
            \texttt{id} & Идентификатор записи \\\hline
            \texttt{skill\_name} & Название навыка, внешний ключ к таблице skills \\\hline
            \texttt{user\_id} & Идентификатор пользователя, внешний ключ к таблице users \\\hline
        \end{tabular}
    \end{center}
\end{table}

Для хранения связей навык-план используется таблица plans-skills:
\begin{table}[!ht]
    \caption{Описание полей таблицы \texttt{plans\_skills}}
    \label{tbl:plans-skills}
    \begin{center}
        \begin{tabular}{|p{0.2\textwidth}p{0.6\textwidth}|}
            \hline
            \textbf{Поле} & \textbf{Значение} \\\hline
            \texttt{id} & Идентификатор записи \\\hline
            \texttt{skill\_name} & Название навыка, внешний ключ к таблице skills \\\hline
            \texttt{plan\_id} & Идентификатор плана, внешний ключ к таблице plans \\\hline
        \end{tabular}
    \end{center}
\end{table}

\newpage
\noindent\textbf{Ролевая модель}\\
В базе данных присутствуют три роли:
\begin{enumerate}
\item Администратор - имеет доступ к всем таблицам, доступны все операции над ними;
\item Пользователь - доступ к всем таблицам для CRUD операций, кроме таблиц telegram\_auth, schema\_migrations и всех таблиц, сявзанных с панелью администратора;
\item Гость - доступ к тем же таблицам, что у пользователя на операции SELECT.
\end{enumerate}

\noindent\textbf{Триггеры}\\
Для обновления поля progress в таблице task, нужен механизм, который будет при каждом обновлении статуса задачи (добавления новой задачи к плану) обновлять поле progress.

Такую задачу может выполнить триггер, который будет срабатывать при обновлении таблицы plan и добавлении новых записей. После таких действий будет срабатывать функция, которая будет актуализировать progress у всех записей таблицы.  

Ниже представлена схема алгоритма работы выполняемой тригером функции update\_progress().

\imgw{update_progress_func}{h!}{0.8\textwidth}{Схема алгоритма работы функции триггера}

\newpage
\section*{Вывод}
В данном разделе:
\begin{itemize}
\item спроектированы сущности базы данных;
\item описаны сущности для миграции данных;
\item спроектирована база данных для хранения сессий;
\item описаны требуемые внешние ключи, триггеры;
\item приведена ролевая модель для разграничения доступа.
\end{itemize}