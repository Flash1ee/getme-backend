\chapter*{Введение}
\addcontentsline{toc}{chapter}{Введение}

Сфера информационных технологий (IT) испытывает нехватку кадров. Каждый год университеты страны выпускают специалистов, но их
количество не удовлетворяет спрос.

Из-за популярности it-специальностей, в последние годы набирают популярность курсы, на которых за небольшой промежуток времени
готовят специалистов в различных направлениях.

Однако существует проблема - как улучшить свои навыки, стать более квалифицированным специалистом, сменить направление.
Так же существует группа людей, которым не удовлетворяют описанные выше способы получения специальности.

Для решения этой проблемы существует еще одна модель: обучение с наставником, которого называют ментором.

Наставник - специалист, работающий в индустрии и знающий, какие навыки востребованы прямо сейчас. Этот человек
способен составить программу и в короткие сроки подготовить подопечного к будущей работе.
Помимо этого, опытные наставники способны поделиться опытом, а также знаниями с людьми, работающими в индустрии и желающими
повысить уровень своей квалификации, перейти на более оплачиваемую должность.

Для того, чтобы найти наставника, нужна платформа.
Существует большое количество социальых сетей и мессенджеров (Telegram, Twitter, VK, WhatsApp, и другие),
которые позволяют находить людей, общаться, однако для описанной модели это не подходит. Нужна более специализированная
плафторма, покрывающая требуемые потребности.

Реализация приложения для обучения требует наличия базы данных и методов взаимодействия с ней.

Цель работы -- реализовать базу данных для приложения по поиску наставника для изучения информационных технологий.

Чтобы достигнуть поставленной цели, требуется решить следующие задачи:

\begin{itemize}
    \item проанализировать варианты представления данных и выбрать подходящий вариант для решения задачи;
    \item проанализировать системы управления базами данных и выбрать подходящую систему для хранения данных;
    \item спроектировать базу данных, описать ее сущности и связи;
    \item реализовать интерфейс для доступа к базе данных;
    \item реализовать программное обеспечение, которое позволит получить доступ к данным.
    \item исследовать производительность базы данных.
\end{itemize}