\documentclass{bmstu}
 
\usepackage{csvsimple}
\usepackage{longtable}
\usepackage{biblatex}
\addbibresource{main.bib}

\lstset{frame=none, tabsize=4}

\lstdefinelanguage{docker}{
  keywords={FROM, RUN, COPY, ADD, ENTRYPOINT, CMD,  ENV, ARG, WORKDIR, EXPOSE, LABEL, USER, VOLUME, STOPSIGNAL, ONBUILD, MAINTAINER},
  keywordstyle=\color{blue}\bfseries,
  identifierstyle=\color{black},
  sensitive=false,
  comment=[l]{\#},
  commentstyle=\color{purple}\ttfamily,
  stringstyle=\color{red}\ttfamily,
  morestring=[b]',
  morestring=[b]"
}

\lstdefinelanguage{docker-compose}{
  keywords={image, environment, ports, container_name, ports, volumes, links},
  keywordstyle=\color{blue}\bfseries,
  identifierstyle=\color{black},
  sensitive=false,
  comment=[l]{\#},
  commentstyle=\color{purple}\ttfamily,
  stringstyle=\color{red}\ttfamily,
  morestring=[b]',
  morestring=[b]"
}

%%%%%%%%%%%%%%%%%%%%%%%%%%%%%%%%%%%%%%%%%%%%%%%%%%%%%%
%%%%%%%%%%% YAML syntax highlighting %%%%%%%%%%%%%%%%%

% http://tex.stackexchange.com/questions/152829/how-can-i-highlight-yaml-code-in-a-pretty-way-with-listings

% here is a macro expanding to the name of the language
% (handy if you decide to change it further down the road)
\newcommand\YAMLcolonstyle{\color{red}\mdseries}
\newcommand\YAMLkeystyle{\color{black}\bfseries}
\newcommand\YAMLvaluestyle{\color{blue}\mdseries}

\makeatletter

\newcommand\language@yaml{yaml}

\expandafter\expandafter\expandafter\lstdefinelanguage
\expandafter{\language@yaml}
{
  keywords={true,false,null,y,n},
  keywordstyle=\color{darkgray}\bfseries,
  basicstyle=\YAMLkeystyle,                                 % assuming a key comes first
  sensitive=false,
  comment=[l]{\#},
  morecomment=[s]{/*}{*/},
  commentstyle=\color{purple}\ttfamily,
  stringstyle=\YAMLvaluestyle\ttfamily,
  moredelim=[l][\color{orange}]{\&},
  moredelim=[l][\color{magenta}]{*},
  moredelim=**[il][\YAMLcolonstyle{:}\YAMLvaluestyle]{:},   % switch to value style at :
  morestring=[b]',
  morestring=[b]",
  literate =    {---}{{\ProcessThreeDashes}}3
                {>}{{\textcolor{red}\textgreater}}1     
                {|}{{\textcolor{red}\textbar}}1 
                {\ -\ }{{\mdseries\ -\ }}3,
}

% switch to key style at EOL
\lst@AddToHook{EveryLine}{\ifx\lst@language\language@yaml\YAMLkeystyle\fi}
\makeatother

\newcommand\ProcessThreeDashes{\llap{\color{cyan}\mdseries-{-}-}}

%%%%%%%%%%% YAML syntax highlighting %%%%%%%%%%%%%%%%%
%%%%%%%%%%%%%%%%%%%%%%%%%%%%%%%%%%%%%%%%%%%%%%%%%%%%%%

\begin{document}

\makecourseworktitle{Информатика и системы управления}{Программное обеспечение ЭВМ и информационные технологии}{Разработка базы данных для приложения по поиску наставника для изучения информационных технологий}
{ИУ7-66Б}{Д.~В.~Варин}{Ю.~М.~Гаврилова}{}{}

\setcounter{page}{3}

\maketableofcontents

\begin{definitions}
	\definition{Ментор}{наставник, человек, у которого есть большой опыт в какой-то профессиональной области}
	\definition{Менти}{подопечный}
	\definition{Персистентность}{возможность долговременного хранения состояния}
    \definition{База данных}{совокупность данных, хранимых в соответствии со схемой данных, манипулирование которыми выполняют в соответствии с правилами средств моделирования данных}
    \definition{Система управления базами данных}{совокупность программных средств, обеспечивающих управление созданием и использованием баз данных}
    \definition{Structured Query Language}{язык структурированных запросов}
    \definition{Not only Structed Query Language}{термин, обозначающий ряд подходов, направленных на реализацию хранилищ баз данных, имеющих существенные отличия от моделей, используемых в традиционных реляционных СУБД с доступом к данным средствами языка SQL}
    \definition{PostgreSQL}{свободная объектно-реляционная система управления базами данных}
    \definition{MySQL}{свободная реляционная система управления базами данных}
    \definition{Oracle Database}{объектно-реляционная система управления базами данных компании Oracle}
    \definition{Redis}{объектно-реляционная система управления базами данных компании Oracle}
    \definition{Tarantool}{объектно-реляционная система управления базами данных компании Oracle}
    \definition{MongoDB}{объектно-реляционная система управления базами данных компании Oracle}
\end{definitions}

\begin{abbreviations}
    \definition{ПО}{Программное обеспечение}
    \definition{SQL}{Structured Query Language}
    \definition{NoSQL}{Not only Structed Query Language}
    \definition{СУБД}{Система управления базами данных "" --- совокупность программных средств, обеспечивающих управление созданием и использованием баз данных}
\end{abbreviations}

\chapter*{Введение}
\addcontentsline{toc}{chapter}{Введение}

Сфера информационных технологий (IT) испытывает нехватку кадров. Каждый год университеты страны выпускают специалистов, но их
количество не удовлетворяет спрос.

Из-за популярности it-специальностей, в последние годы набирают популярность курсы, на которых за небольшой промежуток времени
готовят специалистов в различных направлениях.

На таких курсах существует менторинг - программа поддержки студентов. Ментор - это профессионал, обладающий компетенциями в своей сфере, источник знаний и ответов. Он помогает развиваться в профессиональной жизни своему подопечному — менти. 

Ментор преследует цель - поделиться собственным опытом с менти. Но на образовательных курсах ментор помогает, в основном, в рамках программы, на которой построен курс. 

Другое направление менторинга - составление индивидуального плана развития,  в программах, описанных выше, не покрывается в исчерпывающе. Индивидуальный подход более действеннен для людей, имеющих некоторый опыт в изучаемой области и желающих повысить свой уровень. Например, младший инженер программист хочет повысить свой уровень до старшего. Для этого ему нужен человек, который сможет составить план, состоящий из конкретных шагов, при достижении которых, уровень компетенции возрастет до желаемого уровня.  

Ментор - специалист, работающий в индустрии и знающий, какие навыки востребованы прямо сейчас. Этот человек
способен составить программу и в короткие сроки подготовить менти к будущей работе или повысить уровень квалификации уже работающим специалистам.

Для того, чтобы найти наставника и составить план развития, нужно приложение, предоставляющее данную функциональность. 

Реализация приложения требует наличия базы данных и методов взаимодействия с ней.

Цель работы -- реализовать базу данных для приложения по поиску наставника для изучения информационных технологий.

Чтобы достигнуть поставленной цели, требуется решить следующие задачи:

\begin{itemize}
    \item проанализировать варианты представления данных и выбрать подходящий вариант для решения задачи;
    \item проанализировать системы управления базами данных и выбрать подходящую систему для хранения данных;
    \item спроектировать базу данных, описать ее сущности и связи;
    \item реализовать интерфейс для доступа к базе данных;
    \item реализовать программное обеспечение, которое позволит получить доступ к данным.
    \item исследовать производительность базы данных.
\end{itemize}
\chapter{Аналитическая часть}

В данном разделе описана концепция взаимодействия наставника с подопечным, постановка задачи устройства приложения для обучения.
Представлен анализ способов хранения данных и систем управления базами данных, оптимальных для решения поставленной задачи.
Приведен анализ существующих решений.

\section{Формализация задачи}

Наставник - человек, работающий в индустрии. Он имеет опыт и экспертизу в сферах информационных технологий.
Он предлагает свои услуги по обучению других людей.

Для подопечного необходима возможность найти наставника по навыкам, которые ему необходимо изучить (например, разработка операционных систем), после чего,
выбрав наставника, уведомить его о желании воспользоваться его услугами.

Наставник имеет возможность просматривать список активных заявок на наставничество.
В личном кабинете имеется список активных заявок, среди которых наставник выбирает себе подопечных, исходя из рейтинга подопечных,
нагрузки и возможностей.

В обучении важна последовательность, следование плану, отслеживание прогресса.

После утверждения заявки, у наставника появляется возножность составлять планы развития для подопечного.
План состоит из задач, у которых есть состояние готовности, срок выполнения.
Подопечный должен в установленные сроки и по мере выполнения, отмечать задачи как выполненные.
Данные о выполнении будут использоваться для построения системы мотивации, например, рейтинга, благодаря которому
наставники могут оценивать усердие и мотивированнность подопечных.

Статистика выполнения может использоваться для мотивации подопечного к обучению, подсчета его рейтинга, который влияет на возможность попасть к наставнику:
на фоне остальных заявок, больший рейтинг повысит вероятность выбора подопечного из-за большего усердия и мотивированности.

Наставник предлагает свои услуги:
\begin{itemize}
    \item рассказ о работе в компании (подопечных может использовать информацию для поиска работы);
    \item получение советов, рекомендаций по карьерному росту;
    \item составление планов развития.
\end{itemize}

Подопечный получает возможности:
\begin{itemize}
    \item изучить навыки, требуемые в работе;
    \item повысить профессиональный уровень;
    \item узнать о работе в разных компаниях, разных сферах;
    \item эффективно обучаться под руководством профессионалов.
\end{itemize}

\section{Базы данных и системы управления базами данных}

В задаче хранения информации системы общения и обучения важную роль имеет выбор модели хранения данных.
Для персистентного хранения данных используются базы данных \cite{database}.
Для управления этими базами данных используется системы управления данных -- СУБД \cite{subd}.

\section{Хранение данных}

Система, разрабатываемая в рамках курсового проекта, предполагает собой микросервисное приложение \cite{microservice}, в котороым каждый сервис отвечает за обработку функциональной части домена.

Доступ к системе имеет разграничение по ролям: наставник, подопечный, администратор. При этом, только у одного типа пользователя системы есть доступ к данным, с которыми он может взаимодейстовать.
Администратор имеет доступ к всей системе целиком.

Для хранения данных необходимо использовать строго структурированную и типизированную базу данных, потому что все роли
имеют отношения между собой, доступ к данным разграничен и для получения информации приходится взаимодействовать с функциональными частями,
учитывая эти связи.

Данные в приложении делятся на следующие типы:
\begin{itemize}
    \item данные пользователей;
    \item данные планов обучения;
    \item данные сессий;
    \item данные связей между наставниками и подопечными.
\end{itemize}

\section{Хранение пользовательских данных и данных о планах обучения}

Разрабатываемая система предполагает хранение данных о пользователях, связях между наставниками и подопечными, а также планов развития.
Поэтому необходимо предусмотреть наличие нескольких ролей, пользователей, каждый из которых будет
иметь возможность загружать свои наборы данных, взаимодейстовать с своей частью системы.

\subsection{Классификация баз данных по способу обработки}

По способу обработки базы данных делятся на две группы -- реляционные и нереляционные базы данных. Каждый из двух типов служит для выполнения определенного рода задач.

\subsubsection{Реляционные базы данных (SQL)}

Данные реляционных баз хранятся в виде таблиц и строк, таблицы могут иметь связи с другими таблицами через внешние ключи, таким образом образуя некие отношения.

Реляционные базы данных используют язык SQL. Структура таких баз данных позволяет связывать информацию из разных таблиц с помощью внешних ключей (или индексов), которые используются для уникальной идентификации любого атомарного фрагмента данных в этой таблице. Другие таблицы могут ссылаться на этот внешний ключ, чтобы создать связь между частями данных и частью, на которую указывает внешний ключ.

SQL используют универсальный язык структурированных запросов для определения и обработки данных. Это накладывает определенные ограничения: прежде чем начать обработку, данные надо разместить внутри таблиц и описать.

\subsubsection{Нереляционные базы данных (NoSQL)}

Данные нереляционных баз данных не имеют общего формата. Они могут представляться в виде документов (Mongo \cite{mongodb}, Tarantool \cite{tarantool}), пар ключ-значение (Redis \cite{redis}), графовых представляниях.

Динамические схемы для неструктурированных данных позволяют:

\begin{itemize}
    \item ориентировать информацию на столбцы или документы;
    \item основывать ее на графике;
    \item организовывать в виде хранилища Key-Value;
    \item создавать документы без предварительного определения их структуры, использовать разный синтаксис;
    \item добавлять поля непосредственно в процессе обработки.
\end{itemize}

%\subsubsection{Выбор модели хранения данных для решения задачи}
%
%Для решения задачи SQL базы данных выглядят более применимо в контексте работы по нескольким причинам:
%
%\begin{itemize}
%    \item задача предполагает наличие отношений;
%    \item данные имеют четкую структуру, которую можно представить в виде таблиц;
%\end{itemize}

\subsection{Классификация баз данных по способу хранения}

Базы данных, по способу хранения, делятся на две группы -- строковые и колоночные. Каждый из этих типов служит для выполнения для определенного рода задач.

\subsubsection{Строковые базы данных}

Строковыми базами даных называются такие базы данных, записи которых в памяти представляются построчно.
Строковые баз данных используются в транзакционных системах (англ. OLTP \cite{OLTP}). Для таких систем характерно большое количество коротких транзакций с операциями вставки, обновления и удаления данных - \texttt{INSERT}, \texttt{UPDATE}, \texttt{DELETE}.

Основной упор в системах OLTP делается на очень быструю обработку запросов, поддержание целостности данных в средах с множественным доступом и эффективность, которая измеряется количеством транзакций в секунду.

Схемой, используемой для хранения транзакционных баз данных, является модель сущностей, которая включает в себя запросы, обращающиеся к отдельным записям. Так же, в OLTP-системах есть подробные и текущие данных.\\

\subsubsection{Колоночные базы данных}

Колоночными базами данных называются базы данных, записи которых в памяти представляются по столбцам. Колоночные базы данных используется в аналитических системах (англ. OLAP \cite{OLAP}). OLAP характеризуется низким объемом транзакций, а запросы часто сложны и включают в себя агрегацию. Время отклика для таких систем является мерой эффективности.

OLAP-системы широко используются методами интеллектуального анализа данных. В таких базах есть агрегированные, исторические данные, хранящиеся в многомерных схемах.

\subsection{Выбор модели хранения данных для решения задачи}

Для решения задачи построчное хранение данных преобладает над колоночным хранением по нескольким причинам:

\begin{itemize}
    \item задача предполагает постоянное добавление и изменение данных;
    \item задача предполагает быструю отзывчивость на запросы пользователя;
    \item задача не предполагает выполнения аналитических запросов;
\end{itemize}

%\subsection{Обзор СУБД с построчным хранением}
%
%В данном подразделе буду рассмотрены популярные построчные СУБД, которые могут быть использованы для реализации хранения в разрабатываемом программном продукте.\\
%
%\subsubsection{PostgreSQL}
%
%PostgreSQL \cite{postgresql} -- это свободно распространяемая объектно-реляционная система управления базами данных, наиболее развитая из открытых СУБД в мире \cite{db-engines-rating-relational}.
%
%PostgreSQL предоставляет транзакции со свойствами атомарности, согласованности, изоляции, долговечности (ACID \cite{ACID}), автоматически обновляемые представления, материализованные представления, триггеры, внешние ключи и хранимые процедуры.
%
%Рассматриваемая СУБД управляет параллелизмом с помощью технологии управления многоверсионным параллелизмом (англ. MVCC \cite{MVCC}).
%Эта технология дает каждой транзакции <<снимок>> текущего состояния базы данных, позволяя вносить изменения, не затрагивая другие транзакции. Это в значительной степени устраняет необходимость в блокировках чтения и гарантирует, что база данных поддерживает принципы ACID. \\
%
%\subsubsection{Oracle Database}
%
%Oracle Database -- объектно-реляционная система управления базами данных компании Oracle \cite{oracle}. На данный момент, рассматриваемая СУБД является самой популярной в мире.
%
%Все транзакции Oracle Database соответствуют обладают свойствами ACID, поддерживает триггеры, внешние ключи и хранимые процедуры. Данная СУБД подходит для разнообразных рабочих нагрузок и может использоваться практически в любых задачах. Особенностью Oracle Database является быстрая работа с большими массивами данных.
%
%Oracle Database может использовать один или более методов параллелизма. Сюда входят механизмы блокировки для гарантии монопольного использования таблицы одной транзакцией, методы временных меток, которые разрешают сериализацию транзакций и планирование транзакций на основе проверки достоверности. \\
%
%\subsubsection{MySQL}
%
%MySQL \cite{mysql} -- свободная реляционная система управления базами данных. Разработку и поддержку MySQL осуществляет корпорация Oracle.
%
%Рассматриваемая СУБД имеет два основных механизма хранения данных: InnoDB и myISAM.
%InnoDB полностью полностью совместим с принципами ACID, в отличии от myISAM.
%СУБД MySQL подходит  для использования при разработке веб-приложений, что объясняется очень тесной интеграцией с популярными языками PHP и Perl.
%
%Реализация параллелизма в СУБД MySQL реализовано с помощью механизма блокировок, который обеспечивает одновременный доступ к данным.

%\subsection{Выбор СУБД для решения задачи}
%
%Для решения задачи была выбрана СУБД PostgreSQL, потому что данная СУБД проста в развертывании, является свободно-распространяемым ПО, эффективно по быстродействию и обладает исчерпывающим функционалом.

\section{Хранение данных сессий}

Для хранения состояний системы, в частности, авторизации пользователей, следует использовать базу данных.
Эти данные должны храниться отдельно от пользовательских с целью безопасности и отказоустойчивости, их обработкой занимается отдельный сервис.
Особенность таких данных в том, что у них нет отношений и связей, они хранятся парами key-value.
Главное требование - данная база должна отличаться быстродействием.

Для решения таких задач следует рассмотреть in-memory СУБД.

In-Memory —- это набор концепций хранения данных, когда они сохраняются в оперативной памяти приложения, а диск используется для бэкапа. В классических подходах данные хранятся на диске, а память — в кэше.

%\subsection{Обзор in-memory СУБД}
%
%\subsubsection{Tarantool}
%
%Tarantool \cite{tarantool} -- это платформа in-memory вычислений с гибкой схемой хранения данных для эффективного создания высоконагруженных приложений. Включает себя базу данных и сервер приложений на языке программирования Lua.
%
%Tarantool обладает высокой скоростью работы по сравнению с традиционными СУБД.
%
%Для хранения данных используется кортежи (англ. tuple) данных.
%Кортеж -- это массив не типизированных данных. Кортежи объединяются в спейсы (англ. space), аналоги таблицы из реляционной модели хранения данных.
%Спейс -- коллекция кортежей, кортеж -- коллекция полей.
%
%В рассматриваемой СУБД реализованы два механизма хранения данных.
%Первый хранит все данные в оперативной памяти, а второй на диске. Для каждого спейса можно задавать различный механизм хранения данных.
%
%Каждый спейс должен быть проиндексирован первичным ключом. Кроме того, поддерживается неограниченное количество вторичных ключей. Каждый из ключей может быть составным.
%
%В Tarantool реализован механизм <<снимков>> текущего состояния хранилища и журналирования всех операций, что позволяет восстановить состояние базы данных после ее перезагрузки.\\
%
%\subsubsection{Redis}
%
%Redis \cite{redis} -- резидентная система управлениями базами данных класса NoSQL с открытым исходным кодом. Основной структурой данных, с которой работает Redis является структура типа <<ключ-значение>>. Данная СУБД используется как для хранения данных, так и для реализации кэшей и брокеров сообщений.
%
%Redis хранит данные в оперативной памяти и снабжена механизмом <<снимков>> и журналирования, что обеспечивает постоянное хранение данных. Предоставляются операции для реализации механизма обмена сообщениями в шаблоне <<издатель-подписчик>>: с его помощью приложения могут создавать программные каналы, подписываться на них и помещать в эти каналы сообщения, которые будут получены всеми подписчиками. Существует поддержка репликации данных типа master-slave, транзакций и пакетной обработки комманд.
%
%Все данные Redis хранит в виде словаря, в котором ключи связаны со своими значениями. Ключевое отличие Redis от других хранилищ данных заключается в том, что значения этих ключей не ограничиваются строками. Поддерживаются следующие абстрактные типы данных:
%
%\begin{itemize}
%    \item строки;
%    \item списки;
%    \item множества;
%    \item хеш-таблицы;
%    \item упорядоченные множества.
%\end{itemize}
%
%Тип данных значения определяет, какие операции доступные для него; поддежриваются высокоуровневые операции: например, объединение, разность или сортировка наборов.

%\subsection{Выбор СУБД для решения задачи}
%
%Для хранения сессионных данных была выбрана СУБД Redis, так как она проста в развертывании и переносимости,
%обладает требуемым типом хранения - key-value и хранится в оперативной памяти. Tarantool более функциональная база данных, для решения
%задачи слишком нагружена.

\section{Формализация данных}

\subsection{База данных пользователей, планов развития и отношений между ролями}

Хранение данных пользователей, планов развития и отношений между ролями должны храниться в одной базе данных.
Пользователи должны иметь уникальные идентификаторы, чтобы их можно было однозначно идентифицировать.

Наставники и подопечные связаны отношением N:M, информация об этих связях должна сохраняться.

Каждый план должен хранить идентификатор наставника и подопечного, чтобы идентифицировать роли.

\subsection{База данных сессий}

База данных сессий должна хранить пары значений: \textit{уникальный идентификатор сессии} - \textit{уникальный идентификатор пользователя}.
Данные должны отдаваться быстро и иметь устойчивость к отказу, например, сохраняться на твердотельный накопитель с возможностью восстановления.

\section{Вывод}

В данном разделе:

\begin{itemize}
    \item рассмотрена структура базы данных для системы;
    \item проанализированы способы хранения информации для системы;
    \item проведен анализ СУБД, используемых для решения задачи;
    \item формализованны данные, используемые в системе;
\end{itemize}

%\addcontentsline{toc}{chapter}{Литература}
\bibliography{main}          % имя библиографической базы (bib-файла)

\chapter{Конструкторская часть}
В данном разделе рассмотрено проектирование базы данных и приложения.

\section{Проектирование базы данных}
База данных будет состоять из следующих сущностей:
\begin{enumerate}
\item Users - пользователи;
\item Plans - планы развития;
\item Task - задачи из плана развития;
\item Skills - изучаемые навыки;
\item SimpleAuth - данные аутентификации вида логин-пароль;
\item TelegramAuth - данные о телеграмме пользователя;
\item Status - статус выполнения задачи;
\item Панель администратора - набор сущностей панели администратора
\item Sessions - хранение сессий пользователей
\end{enumerate}

\newpage
\subsection{Сущность Users}
Сущность Users содержит информацию о пользователях:
\begin{table}[!ht]
    \caption{Описание полей таблицы \texttt{users}}
    \label{tbl:users}
    \begin{center}
        \begin{tabular}{|p{0.3\textwidth}p{0.7\textwidth}|}
            \hline
            \textbf{Поле} & \textbf{Значение} \\\hline
            \texttt{id} & Идентификатор пользователя, уникальный. \\\hline
            \texttt{first\_name} & Имя пользователя в системе\\\hline
            \texttt{last\_name} & Фамилия пользователя \\\hline
            \texttt{nickname} & Псевдоним, короткая альтернатива имени \\\hline
            \texttt{about} & Информация о пользователе \\\hline
            \texttt{is\_searchable} & Признак отоюражения пользователя в поиске \\\hline
            \texttt{tg\_tag} & Псевдоним в Telegram \\\hline
            \texttt{created\_at} & Время регистрации пользователя\\\hline
            \texttt{updated\_at} & Время последнего редактирования информации о пользователе \\\hline
        \end{tabular}
    \end{center}
\end{table}
\subsection{Сущность Plans}
Сущность Plans содержит информацию о планах развития:
\begin{table}[!ht]
    \caption{Описание полей таблицы \texttt{plans}}
    \label{tbl:plans}
    \begin{center}
        \begin{tabular}{|p{0.3\textwidth}p{0.7\textwidth}|}
            \hline
            \textbf{Поле} & \textbf{Значение} \\\hline
            \texttt{id} & Идентификатор плана, уникальный. \\\hline
            \texttt{name} & Название плана развития. \\\hline
            \texttt{about} & Описание плана развития\\\hline
            \texttt{is\_active} & Признак выполнения плана \\\hline
            \texttt{progress} & Процент выполнения плана \\\hline
            \texttt{mentor\_id} & Идентификатор ментора, создавшего план \\\hline
            \texttt{mentee\_id} & Идентификатор менти, обучающемуся по плану \\\hline
            \texttt{created\_at} & Время создания плана \\\hline
            \texttt{updated\_at} & Время последнего изменения информации о плане \\\hline
        \end{tabular}
    \end{center}
\end{table}
\newpage
\subsection{Сущность Task}
Сущность Task содержит информацию о задачах, которые прикреплены к плану:
\begin{table}[!ht]
    \caption{Описание полей таблицы \texttt{plans}}
    \label{tbl:task}
    \begin{center}
        \begin{tabular}{|p{0.3\textwidth}p{0.7\textwidth}|}
            \hline
            \textbf{Поле} & \textbf{Значение} \\\hline
            \texttt{id} & Идентификатор задачи, уникальный. \\\hline
            \texttt{name} & Название задачи. \\\hline
            \texttt{description} & Описание задачи \\\hline
            \texttt{deadline} & Крайний срок выполнения \\\hline
            \texttt{status} & Статус задачи \\\hline
            \texttt{plan\_id} & Идентификатор плана развития, к которому привязана задача \\\hline
            \texttt{created\_at} & Время создания задачи\\\hline
        \end{tabular}
    \end{center}
\end{table}
\newpage

\subsection{Сущность Skills}
Сущность Skills содержит информацию о навыках, существующих в системе:
\begin{table}[!ht]
    \caption{Описание полей таблицы \texttt{skills}}
    \label{tbl:skills}
    \begin{center}
        \begin{tabular}{|p{0.3\textwidth}p{0.7\textwidth}|}
            \hline
            \textbf{Поле} & \textbf{Значение} \\\hline
            \texttt{name} & Название навыка. \\\hline
            \texttt{color} & Цвет навыка (для бейджа) \\\hline
        \end{tabular}
    \end{center}
\end{table}
\subsection{Сущность Status}
Сущность Status содержит информацию о навыках, существующих в системе:
\begin{table}[!ht]
    \caption{Описание полей таблицы \texttt{status}}
    \label{tbl:status}
    \begin{center}
        \begin{tabular}{|p{0.3\textwidth}p{0.7\textwidth}|}
            \hline
            \textbf{Поле} & \textbf{Значение} \\\hline
            \texttt{name} & Текстовое название статуса. \\\hline
            \texttt{color} & Цвет статуса (для бейджа) \\\hline
        \end{tabular}
    \end{center}
\end{table}
\newpage
\subsection{Сущности Авторизации}
В приложении есть два способа авторизации: через мессенджер Telegram\cite{telegram} и через пару логин-пароль.  
Сущность SimpleAuth содержит данные авторизации пользователей через логин-пароль:
\begin{table}[!ht]
    \caption{Описание полей таблицы \texttt{users\_simple\_auth}}
    \label{tbl:users_simple_auth}
    \begin{center}
        \begin{tabular}{|p{0.3\textwidth}p{0.7\textwidth}|}
            \hline
            \textbf{Поле} & \textbf{Значение} \\\hline
            \texttt{id} & Идентификатор записи. \\\hline
            \texttt{login} & Логин пользователя. \\\hline
            \texttt{encrypted\_password} & Зашифрованнный пароль \\\hline
            \texttt{user\_id} & Идентификатор пользователя, к которому привязаны данные авторизации \\\hline
        \end{tabular}
    \end{center}
\end{table}

Сущность TelegramAuth содержит данные авторизации пользователей через мессенджер Telegram:
\begin{table}[!ht]
    \caption{Описание полей таблицы \texttt{users\_simple\_auth}}
    \label{tbl:telegram_auth}
    \begin{center}
        \begin{tabular}{|p{0.3\textwidth}p{0.7\textwidth}|}
            \hline
            \textbf{Поле} & \textbf{Значение} \\\hline
            \texttt{tg\_id} & Имя пользователя в telegram. \\\hline
            \texttt{last\_auth} & Время последнего входа в приложение через telegram. \\\hline
            \texttt{user\_id} & Идентификатор пользователя, к которому привязаны данные авторизации \\\hline
            \texttt{created\_at} & Дата первого входа в систему через telegram \\\hline
        \end{tabular}
    \end{center}
\end{table}
\newpage
\subsection{Сущности панели администратора}
Для администрирования базы данных, поддержания ролевой модели будет создан ряд сущностей.
ER-модель сущностей, в нотации Crow’s Foot, представлена ниже:
\imgw{admin-erd}{h!}{1.0\textwidth}{ER диаграмма сущностей панели администратора}

\section{Миграции базы данных}
Миграции базы данных - это система контроля версий базы данных.
Для обновления базы данных необходимо определить сущность, которая будет контролировать текущую версию.

Сущность SchemaMigrations содержит данные о текущей версии базы данных:
\begin{table}[!ht]
    \caption{Описание полей таблицы \texttt{schema\_migrations}}
    \label{tbl:schema_migrations}
    \begin{center}
        \begin{tabular}{|p{0.3\textwidth}p{0.7\textwidth}|}
            \hline
            \textbf{Поле} & \textbf{Значение} \\\hline
            \texttt{version} & Номер текущей версии БД \\\hline
            \texttt{dirty} & Признак успешного обновления БД \\\hline
        \end{tabular}
    \end{center}
\end{table}

\section{Проектирование базы данных сессий}
Для отслеживания "состояния" между клиентом и сервером необходимо хранить данные, которые идентифицирует конкретного пользователя. Как говорилось выше, для таких задач используют in-memory СУБД.  
Такие базы данных не имеют одной структуры (как реляционные), поэтому можно лишь высокоуровнево спроектировать базу.

Для хранения пользовательские сессий требуются следующие поля:
\begin{table}[!ht]
    \caption{Описание полей таблицы \texttt{sessions}}
    \label{tbl:sessions}
    \begin{center}
        \begin{tabular}{|p{0.3\textwidth}p{0.7\textwidth}|}
            \hline
            \textbf{Поле} & \textbf{Значение} \\\hline
            \texttt{session\_id} & Идентификатор сессии, уникальный \\\hline
            \texttt{user\_id} & Идентификатор пользователя, которому принадлежит сессия \\\hline
        \end{tabular}
    \end{center}
\end{table}

\section{Ограничения, связи между сущностями}
Для избежания дублирования данных сущности связаны между собой через внешние ключи.
\subsection{Внешние ключи}
\begin{itemize}
\item В Таблице telegram\_auth поле user\_id ссылается на поле id таблицы users;
\item В Таблице users\_simple\_auth поле user\_id ссылается на поле id таблицы users;
\item В Таблице plans 
	\begin{itemize}
		\item поле mentee\_id ссылается на поле id таблицы users;
		\item поле mentor\_id ссылается на поле id таблицы users;
	\end{itemize}
\item В Таблице status поле color ссылается на поле name таблицы color;
\item В Таблице task 
	\begin{itemize}
		\item поле plan\_id ссылается на поле id таблицы plans;
		\item поле status ссылается на поле name таблицы status;
	\end{itemize}
\item В таблице skills поле color ссылается на поле name таблицы color;
\item В Таблице offers 
	\begin{itemize}
		\item поле mentee\_id ссылается на поле id таблицы users;
		\item поле mentor\_id ссылается на поле id таблицы users;
		\item поле skill\_name ссылается на поле name таблицы skills;
	\end{itemize}
\end{itemize}
Помимо внешних ключей, в базе данных присутствуют связи типа many-to-many, реализующиеся через промежуточные таблицы.  

Для хранения пользовательские связей навык-пользователь используется таблица users-skills:
\begin{table}[!ht]
    \caption{Описание полей таблицы \texttt{users\_skills}}
    \label{tbl:users-skills}
    \begin{center}
        \begin{tabular}{|p{0.3\textwidth}p{0.7\textwidth}|}
            \hline
            \textbf{Поле} & \textbf{Значение} \\\hline
            \texttt{id} & Идентификатор записи \\\hline
            \texttt{skill\_name} & Название навыка, внешний ключ к таблице skills \\\hline
            \texttt{user\_id} & Идентификатор пользователя, внешний ключ к таблице users \\\hline
        \end{tabular}
    \end{center}
\end{table}

Для хранения связей навык-план используется таблица plans-skills:
\begin{table}[!ht]
    \caption{Описание полей таблицы \texttt{plans\_skills}}
    \label{tbl:plans-skills}
    \begin{center}
        \begin{tabular}{|p{0.3\textwidth}p{0.7\textwidth}|}
            \hline
            \textbf{Поле} & \textbf{Значение} \\\hline
            \texttt{id} & Идентификатор записи \\\hline
            \texttt{skill\_name} & Название навыка, внешний ключ к таблице skills \\\hline
            \texttt{plan\_id} & Идентификатор плана, внешний ключ к таблице plans \\\hline
        \end{tabular}
    \end{center}
\end{table}


\subsection{Ролевая модель}
В базе данных присутствуют три роли:
\begin{enumerate}
\item Администратор - имеет доступ к всем таблицам, доступны все операции над ними;
\item Ментор - доступ к всем таблицам для CRUD операций, кроме таблиц users, telegram\_auth, users\_simple\_auth, schema\_migrations, таблицам панели администратора.
\item Менти - доступ аналогичен доступа роли "Ментор", исключая доступ на update. delete для таблиц plans, offers.
\end{enumerate}

\subsection{Триггеры}
Для обновления поля progress в таблице task, нужен механизм, который будет при каждом обновлении записей и добавлении новык строк обновлять поле progress.  

Такую задачу может выполнить триггер, который будет срабатывать при обновлении таблицы plan и добавлении новых записей. После таких действий будет срабатывать функция, которая будет актуализировать progress у всех записей таблицы.  

\section{Вывод}
В данном разделе:
\begin{itemize}
\item спроектированы сущности базы данных;
\item описаны сущности для миграции данных;
\item спроектирована база данных для хранения сессий;
\item описаны требуемые внешние ключи, триггеры;
\item приведена ролевая модель для разграничения доступа.
\end{itemize}
\chapter{Технологическая часть}
В данном разделе объясняется выбор СУБД, определяются средства реализации, тип приложения и интерфейс, который приложения будет предоставлять потребителям.

Кроме того, приводятся детали реализации, листинги кода, примеры работы программы.  

\section{Выбор СУБД}
Подробный обзор разновидностей СУБД был представлен в аналитическом разделе.

Для системы необходимо выбрать СУБД для двух бизнес логики и для хранения сессий.

Для первого была выбрана СУБД postgreSQL по следующим причинам:
\begin{itemize}
\item распространяется бесплатно;
\item наличие опыта работы с СУБД;
\item большое количество драйверов для взаимодействия с базой данных через системы,написанные на определенном языке программирования (в том числе на golang). 
\end{itemize}

Для хранения сессий было выбрано in-memory хранилище Redis по причине наличия опыта работы с данным хранилищем. Это хранилище удовлетворяет необходимым требованием (key-value хранилище), помимо этого, оно не требует дополнительной настройки для начала использования.  

\section{Выбор средства реализации}
Система, взаимодействующая с базой данных, представляет из себя Web-сервер, доступ к которому осуществляется с помощью REST API \cite{rest-api}.  
Для реализации будет использоваться язык программирования Golang\cite{golang}. Этот язык 
создан для разработки микросервисных web приложений. 
В данном случае приложение будет состоять из двух микросервисов: сервис сессий и сервис бизнес-логики. 

Для взаимодействия с базами данных будут использоваться драйвера, написанные для языка golang, предоставляющие интерфейс взаимодействия посредством языка программирования.  

Для реализации REST будет использоваться web фреймворк echo\cite{web-echo}. Панель администратора строится на основе go-admin\cite{go-admin}. Документирование REST API будет осуществляться с помощью swagger\cite{swagger}, который поддерживает протокол openAPI\cite{openapi}. 

Для сборки приложения в готовый продукт был выбран docker-compose\cite{docker-compose} - оркестратор Docker контейнеров\cite{docker}. 

Docker позволяет изоилировать приложение и разворачивать его на любой машине, независимо от установленных зависимостей. Это реализуется благодаря наличию всех требуемых зависимостей внутри контейнера. 

В отличии от виртуальных машин, имеющих хост-ОС и гостевую ОС. Контейнеры  размещаются на одном физическом сервере с операционной системой хоста, которая разделяет их между собой. Совместное использование ОС хоста между контейнерами делает их менее требовательными к мощности компьютера.

Docker-compose связывает контейнеры в одну систему - в отдельных контейнерах будет сообираться два микросервиса приложения, postgresql и redis. 
 
\section{Выбор типа приложения}

Система является серверным приложением, которое принимает запросы клиентов, обрабатываем их и возвращает ответ.

Клиенты могут взаимодействовать с сервером через REST API интерфейс. Это один из популярных подходов к взаимодейтствию в Web среде. 

\section{Детали реализации}
Для создания триггеров, инициализации ролевой модели использовались миграций, которые приведены в листингах ~ref{lst:migration-1}~---~\ref{lst:migration-2}

Приложение должно предоставлять доступ к разрабатываемой базе данных посредством REST API. Для этого будет предстоит разработать API.
%В разрабатываемом API необходимо учесть различные роли пользователей и соответствующий им функционал, описанный диаграммой вариантов использования и представленный на Рисунке~\ref{img:use-case}, и ограничения.

\texttt{REST API} проектируемого приложения представлен в Таблице~\ref{tbl:rest-api}.

\begin{landscape}
\begin{longtable}{|p{0.4\textwidth}|p{0.125\textwidth}|p{0.9\textwidth}|}
    \caption[Описание \texttt{REST API} реализуемого приложения]{Описание \texttt{REST API} реализуемого приложения} \label{tbl:rest}\\

    \hline
        Путь & Метод & Описание \\
    \endfirsthead

    \multicolumn{3}{l}
    {{\tablename\ \thetable{} -- продолжение}} \\\hline 
        Путь & Метод & Описание \\
    \endhead
    
    \multicolumn{3}{|r|}{{Продолжение на следующей странице}} \\ \hline
    \endfoot
    
    \hline \multicolumn{3}{|r|}{{Конец таблицы}} \\ \hline
    \endlastfoot
     \hline
    /api/v1/offers         & GET & Метод для получения списка подавших заявку на менторство менти \\\hline
    /api/v1/offers                & POST & Метод для подачи заявки на занятия с ментором \\\hline
    /api/v1/offer/{offer\_id}/accept    & POST & Метод одобрения заявки на менторство (ментор принимает заявку менти) \\\hline
     /api/v1/offer/{offer\_id}/accept              & DELETE & Метод отклонения заявки на менторство (ментор отклоняет заявку менти)\\\hline
    /api/v1/plans                & GET & Метод получения списка планов развития \\\hline
    /api/v1/plans/{plan\_id}/task               & POST & Метод для создания задачи в плане развития \\\hline
    /api/v1/plans/{plan\_id}/status       & POST & Метод изменения статуса задачи\\\hline
    /api/v1/user          & GET PUT & Методы для взаимодействия с данными пользователя \\\hline
    /api/v1/user/status        & GET PUT & Методы для взаимодействия с статусом пользователя (ментор/менти) \\\hline
    /api/v1/user/{user\_id}         & GET & Метод для получения информации о пользователе по его id \\\hline
    
    
    /api/v1/auth/telegram/register          & GET & Проверка авторизации с использованием мессенджера Telegram \\\hline
    /api/v1/auth/telegram/login  & GET & Метод получения сессии приложения после авторизации через Telegram \\\hline
     /api/v1/auth/simple/register  & POST & Метод для регистрации пользователи с использованием электронной почты и пароля \\\hline
    /api/v1/auth/simple/login  & POST & Метод для входа пользователя в систему с использованием электронной почты и пароля \\\hline
     /api/v1/auth/token  & GET & Метод для получения токена авторизации\\\hline
    /api/v1/logout         & POST & Метод для выхода пользователя из системы \\\hline
    
    /api/v1/skills           & GET & Метод получения всех поддерживаемых навыков \\\hline
    /api/v1/skills/users          & GET & Метод получения всех пользователей с подходящими навыками конфигурациям нейронных сетей\\\hline
\end{longtable}
\end{landscape}

Для сборки частей системы использовались docker контейнеры, конфигурации которых представлены в листингах \ref{lst:pgsql-docker} - \ref{lst:session-service}

Для их развертывания использовался docker-compose. Листинг конфигурации в yaml формате представлен на листингах \ref{lst:docker-compose-1}~--~\ref{lst:docker-compose-2}.

\section{Вывод}

В данном разделе была определена СУБД, которая будет использоваться для решения задачи, выбран тип приложения. Кроме того, определены средства реализации приложения и представлен интерфейс взаимодействия с приложением (API), приведены детали реализации разрабатываемого приложения: создания ролевой модели, триггеров, сборка и развертывание системы.  


%\chapter{Экспериментальная часть}


\section{Вывод}


%\chapter*{Заключение}
\addcontentsline{toc}{chapter}{Заключение} 





\printbibliography

\begin{appendices}
    \chapter{Скрипт инициализации ролевой модели базы данных}

	В Листинге~\ref{lst:migration-1} приведен скрипт создания ролевой модели базы данных, а именно - создания и настройки соответствующих прав доступа для трех ролей:
	\listingfile{init_roles.sql}{migration-1}{SQL}{Скрипт инициализации ролевой модели базы данных. Часть 1}{}

	\newpage
	
    В Листинге~\ref{lst:migration-2} приведен скрипт создания триггера для обновления прогресса выполнения плана.
	\listingfile{init_trigger.sql}{migration-2}{SQL}{Скрипт создания триггера для обновления прогресса выполнения плана.}{}

	\newpage


	\chapter{Сборка приложения}

    В Листинге~\ref{lst:pgsql-docker} представлен скрипт создания docker-образа базы данных.

    \listingfile{postgresql.Dockerfile}{pgsql-docker}{docker}{Dockerfile для базы данных}{}

	 \newpage 
	 
 	В Листингах \ref{lst:main-service} и \ref{lst:session-service} представлены скрипты создания docker-образов микросервисов приложения.

    \listingfile{main-service.Dockerfile}{main-service}{docker}{Dockerfile для сервиса основного приложения}{}

    \listingfile{session-service.Dockerfile}{session-service}{docker}{Dockerfile для сервиса авторизации}{}

	\chapter{Развертывание приложения}
	
	Для развертывания приложения использовался docker-compose. С помощью него запускается приложение, состаящее из docker контейнеров.
	
	 В Листингах~\ref{lst:docker-compose-1}~--~\ref{lst:docker-compose-2} представлена конфигурация развертывания приложения.
	 
	\listingfile{docker-compose.yml}{docker-compose-1}{docker-compose}{Конфигурация развертывания приложения. Часть 1}{linerange={1-35}}
	
	\newpage
	
	\listingfile{docker-compose.yml}{docker-compose-2}{docker-compose}{Конфигурация развертывания приложения. Часть 2}{linerange={36-74},firstnumber=36}
\end{appendices}
% \begin{appendices}
% 	\chapter{Первое приложение}
% 	Текст приложения.
% \end{appendices}

\end{document}
