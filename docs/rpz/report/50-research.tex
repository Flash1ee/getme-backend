\chapter{Экспериментальная часть}
%В данном разделе осуществляется постановка эксперимента, демонстрирующего преимущества и недостатки кеширования ответов веб приложения, находящегося под нагрузкой. Описан эксперимент, приведены технические характеристики устройства, на котором осуществлялось тестирование, приведены результаты тестирования и сформулированы выводы.

\section{Цель эксперимента}

Цель эксперимента - сравнение времени, которое требуется для получения данных о пользователях приложения с использованием кэширования данных и без него.  

Для достижения цели требуется:
\begin{itemize}
\item реализовать в приложении кеширование ответов;
\item выбрать конечную точку, которая будет под нагрузкой и составить запрос;
\item обеспечить наличие данных в базе данных;
\item составить ленту запросов и выбрать профиль нагрузки;
\item нагрузить выбранную конечную точку с использованием кеширования и без использования;
\item проанализировать полученыые результаты (графики, метрики) и сформулирвоать выводы.
\end{itemize}

\section{Описание эксперимента}
Для эксперимента в приложении был реализован механизм кеширования ответов: на каждый запрос проверяются переданные аргументы, после чего в кеше осуществляется поиск ответа. Если ответа нет - происходит обращение к базе данных, данные сохраняются в кеш и отдаются пользователю. 

При повторном запросе данные отдадутся из кеша. Запроса в базу данных не будет. 
За счет этого удается сократить время выполнения запроса, исключив поход в базу данных, который занимает определенное время.  

В качестве кеша подключен экземпляр базы данных redis, чтобы не влиять на redis, использующийся для авторизации.  

Для тестирования будет использоваться инструмент нагрузки и измерения производительности Web приложений - Yandex.Tank\cite{yandex-tank}.  

Для обработки результатов нагрузочного тестирования будет использоваться сервис Overload\cite{yandex-overload}. Он позволяет анализировать приложение под нагрузкой, визуализируя результаты в виде графиков.  

Для тестирования была выбрана конечная точка \textit{/api/v1/skills/users} с 
параметром запроса \textit{skills} равным всем возможным значениям из таблицы \textbf{skills}.  
Это позволит сильнее всего нагрузить приложения.
Для получения всех пользователей придется выбрать с записи из таблицы \textbf{users} и применить фильтр для всех пользователей.  

С помощью миграций реализовано заполнение базы данных.  

Для генератора нагрузки (Yandex.Tank) необходимо выбрать тип нагрузки.  
Существует два типа нагрузки: открытая и закрытая.  

\textbf{\textit{Закрытая нагрузка}} подразумевает управление соединениями, в каждом из которых генератор подает запросы с нулевой задержкой между ними. Это создает эффект, при котором деградация тестируемой системы влечет за собой уменьшение потока запросов (при неизменном количестве соединений). Данный тип нагрузки используется для определения органической производительности тестируемой системы. 
Для нагрузки данного типа необходимо задать число $N$, где $N$ - суммарное количество запросов, которое будет сделано к системе. 
Данный тип нагруки является менее жестким по сравнению с открытой.

\textbf{\textit{Открытая нагрузка}} подразумевает управление количеством одновременных запросов в систему, которое не зависит от текущего состояния тестируемой системы и потенциально является более жесткой по сравнению с закрытой нагрузкой. Необходимо задать количество $rps$, где $rps$ - количество одновременных запросов к приложению в секунду.  

Для тестирования был выбран открытый тип, так как такая нагрузка сильнее влияет на систему, нежели закрытая.  

\newpage

В листинге \ref{lst:load-config} представлена конфигурация для генератора нагрузки.

\listingfile{load.yaml}{load-config}{yaml}{Пример конфигурации для генератор нагрузки Yandex.Tank}{}

\newpage
В листинге \ref{lst:ammo-load} представлена лента запросов, использующаяся для тестирования.

\listingfile{ammo.txt}{ammo-load}{}{Пример ленты запросов}{}

С такой лентой запроса на каждого клиента будет приходиться три запроса к приложению. 

Для тестирования была выбрано два профиля нагрузки:
\begin{enumerate}
	\item линейная - от $1$ до $100$ rps на протяжении пяти минут;
	\item постоянная - $250$ rps на протяжении минуты.
\end{enumerate}



\section{Технические характеристики}
Эксперимент производился с использованием одного компьютера.
Его технические характеристики:
\begin{itemize}
	\item процессор: Apple M1 Pro;
	\item память: 32~Гб;
	\item операционная система: macOS~Monterey\cite{monterey} 12.4.
\end{itemize}

Исследование проводилось на ноутбуке, подключенным к сети электропитания. Во время тестирования ноутбук был нагружен только встроенными приложениями окружения рабочего стола, окружением рабочего стола, а также непосредственно тестируемой системой.

\newpage

\section{Результаты эксперимента}
Ниже приведены графики зависимости времени ответа конечной точки от количества rps, подаваемых генератором нагрузки.  

На графиках показаны метрики rps, avg и 98 квантиль времени ответа.

rps (requests per second) - это количество запросов, подаваемых генератором в секунду.  

avg (average) -  это среднее время ответа конечной точки.  

98 квантиль времени ответа - это значение времени ответа, до которого укладывается 98 \% всех запросов.

\subsection{Линейная нагрузка}
График зависимости времени ответа конечной точки при линейной нагрузке без использования кеширования.   
\imgw{load-line-no-cache}{h!}{0.7\textwidth}{График времени ответа конечной точки при линейной нагрузке без использования кеширования}

\newpage

График зависимости времени ответа конечной точки при линейной нагрузке с использования кеширования.   
\imgw{load-line-with-cache}{h!}{0.7\textwidth}{График времени ответа конечной точки при линейной нагрузке с использованием кеширования}

Тестирование показало, что при заданном количестве запросов (максимум 100 в секунду), кеширование не дает повышение производиельности в сравнении с временем ответа без кеширования. 
 
Наоброт, появляются более частные "пики", связанные с задержками в ожидании запросов из кеша и реляционной базы данных.  

\newpage
\subsection{Постоянная нагрузка}

График зависимости времени ответа конечной точки при постоянной нагрузке без использования кеширования.   
\imgw{load-const-no-cache}{h!}{0.7\textwidth}{График времени ответа конечной точки при постоянной нагрузке без использования кеширования}
\newpage

График зависимости времени ответа конечной точки при постоянной нагрузке с использования кеширования.   
\imgw{load-const-with-cache}{h!}{0.7\textwidth}{График времени ответа конечной точки при постоянной нагрузке с использованием кеширования}

По графикам видно, что кеширование при большой нагрузке (250 запросов в секунду) показывает лучшее время ответа в сравнении с тестированием без кеширования. 
Без кеша среднее время ответа - 1000 миллисекунд.  
С кешом - 15 миллисекунд.  

\newpage

\section*{Вывод}

В ходе эксперимента было реализовано кеширование ответов приложения, выбрана и протестирована конечная точка \textit{/api/v1/skills/users}.

Эксперимент показал, что кеширование позволило уменьшить время ответа конечной точки.\\
При RPS = 250 время сократилось в 66 раз. Но при линейном изменении RPS от 1 до 100 уменьшения времени ответа не было, производительность ухудшилась: появились более частые пики, связанные с долгими ответами баз данных.  

Внедрение кеширования может увеличить производительность на конечную точку при наличии высокой нагрузки.  
Перед добавление кеширования необходимо проанализировать конечную точку на предмет высокого коэффициента попадания запросов в кеш, иначе кеширование не улучшит производительность, а ухудшит ее.  
