\begin{appendices}
    \chapter{Скрипт инициализации ролевой модели базы данных}

	В Листинге~\ref{lst:migration-1} приведен скрипт создания ролевой модели базы данных, а именно - создания и настройки соответствующих прав доступа для трех ролей:
	\listingfile{init_roles.sql}{migration-1}{SQL}{Скрипт инициализации ролевой модели базы данных. Часть 1}{}

	\newpage
	
    В Листинге~\ref{lst:migration-2} приведен скрипт создания триггера для обновления прогресса выполнения плана.
	\listingfile{init_trigger.sql}{migration-2}{SQL}{Скрипт создания триггера для обновления прогресса выполнения плана.}{}

	\newpage


	\chapter{Сборка приложения}

    В Листинге~\ref{lst:pgsql-docker} представлен скрипт создания docker-образа базы данных.

    \listingfile{postgresql.Dockerfile}{pgsql-docker}{docker}{Dockerfile для базы данных}{}

	 \newpage 
	 
 	В Листингах \ref{lst:main-service} и \ref{lst:session-service} представлены скрипты создания docker-образов микросервисов приложения.

    \listingfile{main-service.Dockerfile}{main-service}{docker}{Dockerfile для сервиса основного приложения}{}

    \listingfile{session-service.Dockerfile}{session-service}{docker}{Dockerfile для сервиса авторизации}{}

	\chapter{Развертывание приложения}
	
	Для развертывания приложения использовался docker-compose. С помощью него запускается приложение, состаящее из docker контейнеров.
	
	 В Листингах~\ref{lst:docker-compose-1}~--~\ref{lst:docker-compose-2} представлена конфигурация развертывания приложения.
	 
	\listingfile{docker-compose.yml}{docker-compose-1}{docker-compose}{Конфигурация развертывания приложения. Часть 1}{linerange={1-35}}
	
	\newpage
	
	\listingfile{docker-compose.yml}{docker-compose-2}{docker-compose}{Конфигурация развертывания приложения. Часть 2}{linerange={36-74},firstnumber=36}
\end{appendices}