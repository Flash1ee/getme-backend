\begin{appendices}
    \chapter{Скрипты инициализации объектов БД}

 	В Листингах \ref{lst:create-db-1}-\ref{lst:create-db-8} представлены скрипты создания объектов БД.
    \listingfile{1_init.up.sql}{create-db-1}{SQL}{Скрипт создания объектов БД. Часть 1}{linerange={1-38}}

    \listingfile{1_init.up.sql}{create-db-12}{SQL}{Скрипт создания объектов БД. Часть 2}{linerange={39-86}}
	\clearpage

    \listingfile{1_init.up.sql}{create-db-13}{SQL}{Скрипт создания объектов БД. Часть 3}{linerange={87-110}}

    \listingfile{2_init.up.sql}{create-db-2}{SQL}{Скрипт создания объектов БД. Часть 4}{}

    \listingfile{3_init.up.sql}{create-db-3}{SQL}{Скрипт создания объектов БД. Часть 5}{}

    \listingfile{4_init.up.sql}{create-db-4}{SQL}{Скрипт создания объектов БД. Часть 6}{}
    
	\clearpage
	
    \listingfile{5_init.up.sql}{create-db-5}{SQL}{Скрипт создания объектов БД. Часть 7}{}

    \listingfile{6_init.up.sql}{create-db-6}{SQL}{Скрипт создания объектов БД. Часть 8}{}
	
    \listingfile{7_init.up.sql}{create-db-7}{SQL}{Скрипт создания объектов БД. Часть 9}{}

    \listingfile{8_init.up.sql}{create-db-8}{SQL}{Скрипт создания объектов БД. Часть 10}{}

    \chapter{Паттерны взаимодействия с PostgreSQL}

	В Листингах~\ref{lst:example-postgresql}~--~\ref{lst:example-postgresql-5} приведен пример взаимодействия с PostgreSQL через приложение.

    \listingfile{user_repository.go}{example-postgresql}{Go}{Взаимодействие с PostgreSQL с помощью паттерна репозиторий. Часть 1}{linerange={1-32}}
	\clearpage

    \listingfile{user_repository.go}{example-postgresql-2}{Go}{Взаимодействие с PostgreSQL с помощью паттерна репозиторий. Часть 2}{linerange={33-69}}
	\clearpage

    \listingfile{user_repository.go}{example-postgresql-3}{Go}{Взаимодействие с PostgreSQL с помощью паттерна репозиторий. Часть 3}{linerange={70-102}}
	\clearpage
    \listingfile{user_repository.go}{example-postgresql-4}{Go}{Взаимодействие с PostgreSQL с помощью паттерна репозиторий. Часть 4}{linerange={103-138}}
	\clearpage
    \listingfile{user_repository.go}{example-postgresql-5}{Go}{Взаимодействие с PostgreSQL с помощью паттерна репозиторий. Часть 5}{linerange={139-180}}


    \chapter{Паттерны взаимодействия с Redis}

	В Листингах~\ref{lst:example-redis}~--~\ref{lst:example-redis-2} приведен пример взаимодействия с Redis через приложение.

    \listingfile{redis_repository.go}{example-redis}{Go}{Взаимодействие с Redis в сервисе авторизации. Часть 1}{linerange={1-32}}
	\clearpage
    \listingfile{redis_repository.go}{example-redis-2}{Go}{Взаимодействие с Redis в сервисе авторизации. Часть 2}{linerange={33-80}}


    \chapter{Скрипт заполнения БД}

	В Листингах~\ref{lst:fill-db}~--~\ref{lst:fill-db-3} приведен скрипт с примером заполнения базы данных:

	\listingfile{fill_db.sql}{fill-db}{SQL}{Скрипт с примером заполнения БД. Часть 1}{linerange={1-38}}

    \listingfile{fill_db.sql}{fill-db-2}{SQL}{Скрипт с примером заполнения БД. Часть 2}{linerange={39-75}}
	\clearpage
	\listingfile{fill_db.sql}{fill-db-3}{SQL}{Скрипт с примером заполнения БД. Часть 3}{linerange={76-107}}

    \chapter{Скрипт инициализации ролевой модели базы данных}

	В Листинге~\ref{lst:migration-1} приведен скрипт создания ролевой модели базы данных, а именно - создания и настройки соответствующих прав доступа для трех ролей:
	\listingfile{init_roles.sql}{migration-1}{SQL}{Скрипт инициализации ролевой модели базы данных. Часть 1}{}

	\newpage
	
    В Листинге~\ref{lst:migration-2} приведен скрипт создания триггера для обновления прогресса выполнения плана.
	\listingfile{init_trigger.sql}{migration-2}{SQL}{Скрипт создания триггера для обновления прогресса выполнения плана.}{}

	\newpage


	\chapter{Сборка приложения}

    В Листинге~\ref{lst:pgsql-docker} представлен скрипт создания docker-образа базы данных.

    \listingfile{postgresql.Dockerfile}{pgsql-docker}{docker}{Dockerfile для базы данных}{}

	 \newpage 
	 
 	В Листингах \ref{lst:main-service} и \ref{lst:session-service} представлены скрипты создания docker-образов микросервисов приложения.

    \listingfile{main-service.Dockerfile}{main-service}{docker}{Dockerfile для сервиса основного приложения}{}

    \listingfile{session-service.Dockerfile}{session-service}{docker}{Dockerfile для сервиса авторизации}{}

	\chapter{Развертывание приложения}
	
	Для развертывания приложения использовался docker-compose. С помощью него запускается приложение, состаящее из docker контейнеров.
	
	 В Листингах~\ref{lst:docker-compose-1}~--~\ref{lst:docker-compose-2} представлена конфигурация развертывания приложения.
	 
	\listingfile{docker-compose.yml}{docker-compose-1}{docker-compose}{Конфигурация развертывания приложения. Часть 1}{linerange={1-35}}
	
	\newpage
	
	\listingfile{docker-compose.yml}{docker-compose-2}{docker-compose}{Конфигурация развертывания приложения. Часть 2}{linerange={36-74},firstnumber=36}
\end{appendices}