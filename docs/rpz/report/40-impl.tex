\chapter{Технологическая часть}
%В данном разделе объясняется выбор СУБД, определяются средства реализации, тип приложения и интерфейс, который приложения будет предоставлять потребителям.
%
%Кроме того, приводятся детали реализации, листинги кода, примеры работы программы.  

\section{Выбор СУБД}
Подробный обзор разновидностей СУБД был представлен в аналитическом разделе.

Для системы необходимо выбрать СУБД для двух бизнес логики и для хранения сессий.

Для первого была выбрана СУБД postgreSQL по следующим причинам:
\begin{itemize}
\item распространяется бесплатно;
\item наличие опыта работы с СУБД;
\item большое количество драйверов для взаимодействия с базой данных через системы,написанные на определенном языке программирования (в том числе на golang). 
\end{itemize}

Для хранения сессий было выбрано in-memory хранилище Redis по причине наличия опыта работы с данным хранилищем. Это хранилище удовлетворяет необходимым требованием (key-value хранилище), помимо этого, оно не требует дополнительной настройки для начала использования.  

\section{Выбор средства реализации}
Система, взаимодействующая с базой данных, представляет из себя Web-сервер, доступ к которому осуществляется с помощью REST API \cite{rest-api}.  
Для реализации будет использоваться язык программирования Golang\cite{golang}. Этот язык 
создан для разработки микросервисных web приложений. 
В данном случае приложение будет состоять из двух микросервисов: сервис сессий и сервис бизнес-логики. 

Для взаимодействия с базами данных будут использоваться драйвера, написанные для языка golang, предоставляющие интерфейс взаимодействия посредством языка программирования.  

Для реализации REST будет использоваться web фреймворк echo\cite{web-echo}. Панель администратора строится на основе go-admin\cite{go-admin}. Документирование REST API будет осуществляться с помощью swagger\cite{swagger}, который поддерживает протокол openAPI\cite{openapi}. 

Для сборки приложения в готовый продукт был выбран docker-compose\cite{docker-compose} - оркестратор Docker контейнеров\cite{docker}. 

Docker позволяет изоилировать приложение и разворачивать его на любой машине, независимо от установленных зависимостей. Это реализуется благодаря наличию всех требуемых зависимостей внутри контейнера. 

В отличии от виртуальных машин, имеющих хост-ОС и гостевую ОС. Контейнеры  размещаются на одном физическом сервере с операционной системой хоста, которая разделяет их между собой. Совместное использование ОС хоста между контейнерами делает их менее требовательными к мощности компьютера.

Docker-compose связывает контейнеры в одну систему - в отдельных контейнерах будет сообираться два микросервиса приложения, postgresql и redis. 
 
\section{Детали реализации}

\textbf{Выбор типа приложения}\\
Система является серверным приложением, которое принимает запросы клиентов, обрабатываем их и возвращает ответ.

Клиенты могут взаимодействовать с сервером через REST API интерфейс. Это один из популярных подходов к взаимодейтствию в Web среде. 

Для создания триггеров, инициализации ролевой модели использовались миграций, которые приведены в листингах ~\ref{lst:migration-1}~---~\ref{lst:migration-2}

Приложение должно предоставлять доступ к разрабатываемой базе данных посредством REST API. Для этого будет предстоит разработать API.
%В разрабатываемом API необходимо учесть различные роли пользователей и соответствующий им функционал, описанный диаграммой вариантов использования и представленный на Рисунке~\ref{img:use-case}, и ограничения.

\texttt{REST API} проектируемого приложения представлен в Таблице~\ref{tbl:rest-api}.

\begin{landscape}
\begin{longtable}{|p{0.4\textwidth}|p{0.125\textwidth}|p{0.9\textwidth}|}
    \caption[Описание \texttt{REST API} реализуемого приложения]{Описание \texttt{REST API} реализуемого приложения}\\
	\label{tbl:rest-api}\\
    \hline
        Путь & Метод & Описание \\
    \endfirsthead

    \multicolumn{3}{l}
    {{\tablename\ \thetable{} -- продолжение}} \\\hline 
        Путь & Метод & Описание \\
    \endhead
    
    \multicolumn{3}{|r|}{{Продолжение на следующей странице}} \\ \hline
    \endfoot
    
    \hline \multicolumn{3}{|r|}{{Конец таблицы}} \\ \hline
    \endlastfoot
     \hline
    /api/v1/offers         & GET & Метод для получения списка подавших заявку на менторство менти \\\hline
    /api/v1/offers                & POST & Метод для подачи заявки на занятия с ментором \\\hline
    /api/v1/offer/{offer\_id}/accept    & POST & Метод одобрения заявки на менторство (ментор принимает заявку менти) \\\hline
     /api/v1/offer/{offer\_id}/accept              & DELETE & Метод отклонения заявки на менторство (ментор отклоняет заявку менти)\\\hline
    /api/v1/plans                & GET & Метод получения списка планов развития \\\hline
    /api/v1/plans/{plan\_id}/task               & POST & Метод для создания задачи в плане развития \\\hline
    /api/v1/plans/{plan\_id}/status       & POST & Метод изменения статуса задачи\\\hline
    /api/v1/user          & GET PUT & Методы для взаимодействия с данными пользователя \\\hline
    /api/v1/user/status        & GET PUT & Методы для взаимодействия с статусом пользователя (ментор/менти) \\\hline
    /api/v1/user/{user\_id}         & GET & Метод для получения информации о пользователе по его id \\\hline
    
    
    /api/v1/auth/telegram/register          & GET & Проверка авторизации с использованием мессенджера Telegram \\\hline
    /api/v1/auth/telegram/login  & GET & Метод получения сессии приложения после авторизации через Telegram \\\hline
     /api/v1/auth/simple/register  & POST & Метод для регистрации пользователи с использованием электронной почты и пароля \\\hline
    /api/v1/auth/simple/login  & POST & Метод для входа пользователя в систему с использованием электронной почты и пароля \\\hline
     /api/v1/auth/token  & GET & Метод для получения токена авторизации\\\hline
    /api/v1/logout         & POST & Метод для выхода пользователя из системы \\\hline
    
    /api/v1/skills           & GET & Метод получения всех поддерживаемых навыков \\\hline
    /api/v1/skills/users          & GET & Метод получения всех пользователей с подходящими навыками конфигурациям нейронных сетей\\\hline
\end{longtable}
\end{landscape}

Для сборки частей системы использовались docker контейнеры, конфигурации которых представлены в листингах \ref{lst:pgsql-docker} - \ref{lst:session-service}

Для их развертывания использовался docker-compose. Листинг конфигурации в yaml формате представлен на листингах \ref{lst:docker-compose-1}~--~\ref{lst:docker-compose-2}.

\section*{Вывод}
В данном разделе была определена СУБД, которая будет использоваться для решения задачи, выбран тип приложения. Кроме того, определены средства реализации приложения и представлен интерфейс взаимодействия с приложением (API), приведены детали реализации разрабатываемого приложения: создания ролевой модели, триггеров, сборка и развертывание системы.  

