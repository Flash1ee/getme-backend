\chapter*{ЗАКЛЮЧЕНИЕ}
\addcontentsline{toc}{chapter}{ЗАКЛЮЧЕНИЕ} 

При выполнении курсовой работы была достигнута ее цель --- спроектирована и разработана база данных для ля приложения по поиску наставника для изучения информационных технологий.
    
В процессе достижения данной цели были решены следующие задачи:

\begin{itemize}
	\item проанализированы варианты представления данных и выбраны подходящие для решения задачи;
    \item проанализированы системы управления базами данных и выбраны подходящие для хранения данных;
    \item спроектирована базу данных, описаны ее сущности и связи;
    \item реализован интерфейс для доступа к базе данных;
    \item реализовано программное обеспечение, позволяющее взаимодействовать со спроектированной базой данных;
\end{itemize}

В ходе курсовой работы были получены знания в области проектирования баз данных, кеширования данных и нагрузочного приложения, предоставляющего интерфейс взаимодействия с базой данных.  

Были изучены различные типы баз данных, способы хранения данных.  

В результате проделанной работы, было разработано программное обеспечение, предоставляющее интерфейс для работы с базой данных. Была увеличена производительность с помощью внедрения механизма кеширования ответов.

В ходе выполнения исследовательской части работы было установлено, что кэширование данных при высокой нагрузке (250 запросов в секунду) приводит к росту производительности системы с точки зрения времени ответа.
Время ответа уменьшилось в 66 раз по сравнению с вариантом без кэширования.  
 
