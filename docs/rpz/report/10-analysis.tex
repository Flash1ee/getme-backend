\chapter{Аналитическая часть}

В данном разделе описана концепция взаимодействия наставника с подопечным, постановка задачи устройства приложения для обучения.
Представлен анализ способов хранения данных и систем управления базами данных, оптимальных для решения поставленной задачи.
Приведен анализ существующих решений.

\section{Формализация задачи}

Наставник - человек, работающий в индустрии. Он имеет опыт и экспертизу в сферах информационных технологий.
Он предлагает свои услуги по обучению других людей.

Для подопечного необходима возможность найти наставника по навыкам, которые ему необходимо изучить (например, разработка операционных систем), после чего,
выбрав наставника, уведомить его о желании воспользоваться его услугами.

Наставник имеет возможность просматривать список активных заявок на наставничество.
В личном кабинете имеется список активных заявок, среди которых наставник выбирает себе подопечных, исходя из рейтинга подопечных,
нагрузки и возможностей.

В обучении важна последовательность, следование плану, отслеживание прогресса.

После утверждения заявки, у наставника появляется возножность составлять планы развития для подопечного.
План состоит из задач, у которых есть состояние готовности, срок выполнения.
Подопечный должен в установленные сроки и по мере выполнения, отмечать задачи как выполненные.
Данные о выполнении будут использоваться для построения системы мотивации, например, рейтинга, благодаря которому
наставники могут оценивать усердие и мотивированнность подопечных.

Статистика выполнения может использоваться для мотивации подопечного к обучению, подсчета его рейтинга, который влияет на возможность попасть к наставнику:
на фоне остальных заявок, больший рейтинг повысит вероятность выбора подопечного из-за большего усердия и мотивированности.

Наставник предлагает свои услуги:
\begin{itemize}
    \item рассказ о работе в компании (подопечных может использовать информацию для поиска работы);
    \item получение советов, рекомендаций по карьерному росту;
    \item составление планов развития.
\end{itemize}

Подопечный получает возможности:
\begin{itemize}
    \item изучить навыки, требуемые в работе;
    \item повысить профессиональный уровень;
    \item узнать о работе в разных компаниях, разных сферах;
    \item эффективно обучаться под руководством профессионалов.
\end{itemize}

\section{Базы данных и системы управления базами данных}

В задаче хранения информации системы общения и обучения важную роль имеет выбор модели хранения данных.
Для персистентного хранения данных используются базы данных \cite{database}.
Для управления этими базами данных используется системы управления данных -- СУБД \cite{subd}.

\section{Хранение данных}

Система, разрабатываемая в рамках курсового проекта, предполагает собой микросервисное приложение \cite{microservice}, в котороым каждый сервис отвечает за обработку функциональной части домена.

Доступ к системе имеет разграничение по ролям: наставник, подопечный, администратор. При этом, только у одного типа пользователя системы есть доступ к данным, с которыми он может взаимодейстовать.
Администратор имеет доступ к всей системе целиком.

Для хранения данных необходимо использовать строго структурированную и типизированную базу данных, потому что все роли
имеют отношения между собой, доступ к данным разграничен и для получения информации приходится взаимодействовать с функциональными частями,
учитывая эти связи.

Данные в приложении делятся на следующие типы:
\begin{itemize}
    \item данные пользователей;
    \item данные планов обучения;
    \item данные сессий;
    \item данные связей между наставниками и подопечными.
\end{itemize}

\section{Хранение пользовательских данных и данных о планах обучения}

Разрабатываемая система предполагает хранение данных о пользователях, связях между наставниками и подопечными, а также планов развития.
Поэтому необходимо предусмотреть наличие нескольких ролей, пользователей, каждый из которых будет
иметь возможность загружать свои наборы данных, взаимодейстовать с своей частью системы.

\subsection{Классификация баз данных по способу обработки}

По способу обработки базы данных делятся на две группы -- реляционные и нереляционные базы данных. Каждый из двух типов служит для выполнения определенного рода задач.

\subsubsection{Реляционные базы данных (SQL)}

Данные реляционных баз хранятся в виде таблиц и строк, таблицы могут иметь связи с другими таблицами через внешние ключи, таким образом образуя некие отношения.

Реляционные базы данных используют язык SQL. Структура таких баз данных позволяет связывать информацию из разных таблиц с помощью внешних ключей (или индексов), которые используются для уникальной идентификации любого атомарного фрагмента данных в этой таблице. Другие таблицы могут ссылаться на этот внешний ключ, чтобы создать связь между частями данных и частью, на которую указывает внешний ключ.

SQL используют универсальный язык структурированных запросов для определения и обработки данных. Это накладывает определенные ограничения: прежде чем начать обработку, данные надо разместить внутри таблиц и описать.

\subsubsection{Нереляционные базы данных (NoSQL)}

Данные нереляционных баз данных не имеют общего формата. Они могут представляться в виде документов (Mongo \cite{mongodb}, Tarantool \cite{tarantool}), пар ключ-значение (Redis \cite{redis}), графовых представляниях.

Динамические схемы для неструктурированных данных позволяют:

\begin{itemize}
    \item ориентировать информацию на столбцы или документы;
    \item основывать ее на графике;
    \item организовывать в виде хранилища Key-Value;
    \item создавать документы без предварительного определения их структуры, использовать разный синтаксис;
    \item добавлять поля непосредственно в процессе обработки.
\end{itemize}

%\subsubsection{Выбор модели хранения данных для решения задачи}
%
%Для решения задачи SQL базы данных выглядят более применимо в контексте работы по нескольким причинам:
%
%\begin{itemize}
%    \item задача предполагает наличие отношений;
%    \item данные имеют четкую структуру, которую можно представить в виде таблиц;
%\end{itemize}

\subsection{Классификация баз данных по способу хранения}

Базы данных, по способу хранения, делятся на две группы -- строковые и колоночные. Каждый из этих типов служит для выполнения для определенного рода задач.

\subsubsection{Строковые базы данных}

Строковыми базами даных называются такие базы данных, записи которых в памяти представляются построчно.
Строковые баз данных используются в транзакционных системах (англ. OLTP \cite{OLTP}). Для таких систем характерно большое количество коротких транзакций с операциями вставки, обновления и удаления данных - \texttt{INSERT}, \texttt{UPDATE}, \texttt{DELETE}.

Основной упор в системах OLTP делается на очень быструю обработку запросов, поддержание целостности данных в средах с множественным доступом и эффективность, которая измеряется количеством транзакций в секунду.

Схемой, используемой для хранения транзакционных баз данных, является модель сущностей, которая включает в себя запросы, обращающиеся к отдельным записям. Так же, в OLTP-системах есть подробные и текущие данных.\\

\subsubsection{Колоночные базы данных}

Колоночными базами данных называются базы данных, записи которых в памяти представляются по столбцам. Колоночные базы данных используется в аналитических системах (англ. OLAP \cite{OLAP}). OLAP характеризуется низким объемом транзакций, а запросы часто сложны и включают в себя агрегацию. Время отклика для таких систем является мерой эффективности.

OLAP-системы широко используются методами интеллектуального анализа данных. В таких базах есть агрегированные, исторические данные, хранящиеся в многомерных схемах.

\subsection{Выбор модели хранения данных для решения задачи}

Для решения задачи построчное хранение данных преобладает над колоночным хранением по нескольким причинам:

\begin{itemize}
    \item задача предполагает постоянное добавление и изменение данных;
    \item задача предполагает быструю отзывчивость на запросы пользователя;
    \item задача не предполагает выполнения аналитических запросов;
\end{itemize}

%\subsection{Обзор СУБД с построчным хранением}
%
%В данном подразделе буду рассмотрены популярные построчные СУБД, которые могут быть использованы для реализации хранения в разрабатываемом программном продукте.\\
%
%\subsubsection{PostgreSQL}
%
%PostgreSQL \cite{postgresql} -- это свободно распространяемая объектно-реляционная система управления базами данных, наиболее развитая из открытых СУБД в мире \cite{db-engines-rating-relational}.
%
%PostgreSQL предоставляет транзакции со свойствами атомарности, согласованности, изоляции, долговечности (ACID \cite{ACID}), автоматически обновляемые представления, материализованные представления, триггеры, внешние ключи и хранимые процедуры.
%
%Рассматриваемая СУБД управляет параллелизмом с помощью технологии управления многоверсионным параллелизмом (англ. MVCC \cite{MVCC}).
%Эта технология дает каждой транзакции <<снимок>> текущего состояния базы данных, позволяя вносить изменения, не затрагивая другие транзакции. Это в значительной степени устраняет необходимость в блокировках чтения и гарантирует, что база данных поддерживает принципы ACID. \\
%
%\subsubsection{Oracle Database}
%
%Oracle Database -- объектно-реляционная система управления базами данных компании Oracle \cite{oracle}. На данный момент, рассматриваемая СУБД является самой популярной в мире.
%
%Все транзакции Oracle Database соответствуют обладают свойствами ACID, поддерживает триггеры, внешние ключи и хранимые процедуры. Данная СУБД подходит для разнообразных рабочих нагрузок и может использоваться практически в любых задачах. Особенностью Oracle Database является быстрая работа с большими массивами данных.
%
%Oracle Database может использовать один или более методов параллелизма. Сюда входят механизмы блокировки для гарантии монопольного использования таблицы одной транзакцией, методы временных меток, которые разрешают сериализацию транзакций и планирование транзакций на основе проверки достоверности. \\
%
%\subsubsection{MySQL}
%
%MySQL \cite{mysql} -- свободная реляционная система управления базами данных. Разработку и поддержку MySQL осуществляет корпорация Oracle.
%
%Рассматриваемая СУБД имеет два основных механизма хранения данных: InnoDB и myISAM.
%InnoDB полностью полностью совместим с принципами ACID, в отличии от myISAM.
%СУБД MySQL подходит  для использования при разработке веб-приложений, что объясняется очень тесной интеграцией с популярными языками PHP и Perl.
%
%Реализация параллелизма в СУБД MySQL реализовано с помощью механизма блокировок, который обеспечивает одновременный доступ к данным.

%\subsection{Выбор СУБД для решения задачи}
%
%Для решения задачи была выбрана СУБД PostgreSQL, потому что данная СУБД проста в развертывании, является свободно-распространяемым ПО, эффективно по быстродействию и обладает исчерпывающим функционалом.

\section{Хранение данных сессий}

Для хранения состояний системы, в частности, авторизации пользователей, следует использовать базу данных.
Эти данные должны храниться отдельно от пользовательских с целью безопасности и отказоустойчивости, их обработкой занимается отдельный сервис.
Особенность таких данных в том, что у них нет отношений и связей, они хранятся парами key-value.
Главное требование - данная база должна отличаться быстродействием.

Для решения таких задач следует рассмотреть in-memory СУБД.

In-Memory —- это набор концепций хранения данных, когда они сохраняются в оперативной памяти приложения, а диск используется для бэкапа. В классических подходах данные хранятся на диске, а память — в кэше.

%\subsection{Обзор in-memory СУБД}
%
%\subsubsection{Tarantool}
%
%Tarantool \cite{tarantool} -- это платформа in-memory вычислений с гибкой схемой хранения данных для эффективного создания высоконагруженных приложений. Включает себя базу данных и сервер приложений на языке программирования Lua.
%
%Tarantool обладает высокой скоростью работы по сравнению с традиционными СУБД.
%
%Для хранения данных используется кортежи (англ. tuple) данных.
%Кортеж -- это массив не типизированных данных. Кортежи объединяются в спейсы (англ. space), аналоги таблицы из реляционной модели хранения данных.
%Спейс -- коллекция кортежей, кортеж -- коллекция полей.
%
%В рассматриваемой СУБД реализованы два механизма хранения данных.
%Первый хранит все данные в оперативной памяти, а второй на диске. Для каждого спейса можно задавать различный механизм хранения данных.
%
%Каждый спейс должен быть проиндексирован первичным ключом. Кроме того, поддерживается неограниченное количество вторичных ключей. Каждый из ключей может быть составным.
%
%В Tarantool реализован механизм <<снимков>> текущего состояния хранилища и журналирования всех операций, что позволяет восстановить состояние базы данных после ее перезагрузки.\\
%
%\subsubsection{Redis}
%
%Redis \cite{redis} -- резидентная система управлениями базами данных класса NoSQL с открытым исходным кодом. Основной структурой данных, с которой работает Redis является структура типа <<ключ-значение>>. Данная СУБД используется как для хранения данных, так и для реализации кэшей и брокеров сообщений.
%
%Redis хранит данные в оперативной памяти и снабжена механизмом <<снимков>> и журналирования, что обеспечивает постоянное хранение данных. Предоставляются операции для реализации механизма обмена сообщениями в шаблоне <<издатель-подписчик>>: с его помощью приложения могут создавать программные каналы, подписываться на них и помещать в эти каналы сообщения, которые будут получены всеми подписчиками. Существует поддержка репликации данных типа master-slave, транзакций и пакетной обработки комманд.
%
%Все данные Redis хранит в виде словаря, в котором ключи связаны со своими значениями. Ключевое отличие Redis от других хранилищ данных заключается в том, что значения этих ключей не ограничиваются строками. Поддерживаются следующие абстрактные типы данных:
%
%\begin{itemize}
%    \item строки;
%    \item списки;
%    \item множества;
%    \item хеш-таблицы;
%    \item упорядоченные множества.
%\end{itemize}
%
%Тип данных значения определяет, какие операции доступные для него; поддежриваются высокоуровневые операции: например, объединение, разность или сортировка наборов.

%\subsection{Выбор СУБД для решения задачи}
%
%Для хранения сессионных данных была выбрана СУБД Redis, так как она проста в развертывании и переносимости,
%обладает требуемым типом хранения - key-value и хранится в оперативной памяти. Tarantool более функциональная база данных, для решения
%задачи слишком нагружена.

\section{Формализация данных}

\subsection{База данных пользователей, планов развития и отношений между ролями}

Хранение данных пользователей, планов развития и отношений между ролями должны храниться в одной базе данных.
Пользователи должны иметь уникальные идентификаторы, чтобы их можно было однозначно идентифицировать.

Наставники и подопечные связаны отношением N:M, информация об этих связях должна сохраняться.

Каждый план должен хранить идентификатор наставника и подопечного, чтобы идентифицировать роли.

\subsection{База данных сессий}

База данных сессий должна хранить пары значений: \textit{уникальный идентификатор сессии} - \textit{уникальный идентификатор пользователя}.
Данные должны отдаваться быстро и иметь устойчивость к отказу, например, сохраняться на твердотельный накопитель с возможностью восстановления.

\section{Вывод}

В данном разделе:

\begin{itemize}
    \item рассмотрена структура базы данных для системы;
    \item проанализированы способы хранения информации для системы;
    \item проведен анализ СУБД, используемых для решения задачи;
    \item формализованны данные, используемые в системе;
\end{itemize}
