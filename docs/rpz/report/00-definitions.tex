\begin{definitions}
	\definition{Ментор}{наставник, человек, у которого есть большой опыт в какой-то профессиональной области}
	\definition{Менти}{подопечный}
	\definition{Персистентность}{возможность долговременного хранения состояния}
    \definition{База данных}{совокупность данных, хранимых в соответствии со схемой данных, манипулирование которыми выполняют в соответствии с правилами средств моделирования данных}
    \definition{Система управления базами данных}{совокупность программных средств, обеспечивающих управление созданием и использованием баз данных}
    \definition{Structured Query Language}{язык структурированных запросов}
    \definition{Not only Structed Query Language}{термин, обозначающий ряд подходов, направленных на реализацию хранилищ баз данных, имеющих существенные отличия от моделей, используемых в традиционных реляционных СУБД с доступом к данным средствами языка SQL}
    \definition{PostgreSQL}{свободная объектно-реляционная система управления базами данных}
    \definition{MySQL}{свободная реляционная система управления базами данных}
    \definition{Oracle Database}{объектно-реляционная система управления базами данных компании Oracle}
    \definition{Redis}{объектно-реляционная система управления базами данных компании Oracle}
    \definition{Tarantool}{объектно-реляционная система управления базами данных компании Oracle}
    \definition{MongoDB}{объектно-реляционная система управления базами данных компании Oracle}
    \definition{Метрика программного обеспечения}{мера, позволяющая получить численное значение некоторого свойства программного обеспечения или его спецификаций}
\end{definitions}